\documentclass{article}

\usepackage{caption}
\usepackage{amsmath}
\usepackage{proof}
\usepackage[T1]{fontenc}
\usepackage{ebproof}
\usepackage{amssymb} % To provide the \varnothing symbol
\usepackage[latin1]{inputenc} % acentos sin codigo
\usepackage{verbatim} % comentarios
\usepackage[spanish]{babel}
\usepackage{mathtools}
\usepackage{setspace}
\usepackage{soul}%sombreado
\usepackage[pdftex]{color}
\usepackage{semantic}
\usepackage{float}
\usepackage{definitions}
\usepackage{array}
\usepackage{longtable}
\usepackage{xcolor,colortbl}
\usepackage{multirow}
\usepackage[letterpaper]{geometry}
\usepackage{amssymb}

\newcommand{\nvector}[2][a]{#1_{1},#1_{2},#1_{3},\cdots #1_{#2}}
\newcommand{\vect}{(x_1,x_2,\dots,x_n)}
\newcommand\gij[3]{\Gamma \vdash #1 \Rightarrow #2 : #3 }
\newcommand\rulename[1]{\mathrm{(#1)}}
\newcommand{\nothing}{\varnothing} % different from \emptyset
\newcommand{\tto}{\longrightarrow}
\newcommand{\env}{{\Gamma | \Sigma \ \vdash}}
\newcommand{\envE}{{\varnothing | \Sigma \ \vdash}}
\newcommand{\envEP}{{\Gamma | \Sigma \ \vdash \ }}
\newcommand{\envEPP}{{\Gamma | \Sigma' \ \vdash \ }}
\newcommand{\lambdax}{\lambda x}
\newcommand\myeq{\stackrel{\mathclap{\normalfont\mbox{def}}}{=}}
\begin{document}

\begin{enumerate}

\item (16.2.5) By induction on declarative typing derivations. Proceed by cases on the final rule in the derivation.
\begin{itemize}
\item $Case \ T-Var$: $t = x$ and $\Gamma(x) = T$. Immediate, by $TA-Var$.
\item $Case \ T-Abs$: $t = \lambda x:T_1.t_2$, $\Gamma, x:T_1 \vdash \ t_2 : T_2$ and $T = T_1 \to T_2$. By the induction hypothesis, $\Gamma, x:T_1 \vdash \ t_2 : S_2$, for some $S_2 <: T_2$. By $TA-Abs$, $\Gamma  t : T_1 \to S_2$.
\end{itemize}

\begin{comment}
 By S-Arrow, T1!S2 <: T1!T2, as required.
Case T-App: t = t1 t2 ?? t1 : T11!T12 ?? ` t2 : T11 T = T12
By the induction hypothesis, ?? `? t1 : S1 for some S1 <: T11!T12 and ?? `?
t2 : S2 for some S2 <: T11. By the inversion lemma for the subtype relation
(15.3.2), S1 must have the form S11!S12, for some S11 and S12 with T11 <: S11
and S12 <: T12. By transitivity, S2 <: S11. By the completeness of algorithmic
subtyping, `? S2 <: S11. Now, by TA-App, ?? `? t1 t2 : S12, which finishes this
case (since we already have S12 <: T12).
Case T-Rcd: t = {li=ti
i21..n} ?? ` ti : Ti for each i
T = {li:Ti
i21..n}
Straightforward.
Case T-Proj: t = t1.lj ?? ` t1 : {li:Ti
i21..n} T = Tj
Similar to the application case.
Case T-Sub: t : S S <: T
By the induction hypothesis and transitivity of subtyping.
\end{comment}
\item (16.2.6) If we dropped the arrow subtyping rule $S-Arrow$ but kept the rest of the declarative subtyping and typing rules the same, the system do not still have the minimal typing property. For example, the term $\lambda x:\{a:Nat\}.x$ has both the types $\{a:Nat\} \to \{a:Nat\}$ and $\{a:Nat\} \to Top$ under the declarative rules. With the algorithmic typing has the type $\{a:Nat\} \to \{a:Nat\}$,  but without $S-Arrow$, this type is incomparable with $\{a:Nat\} \to Top$.
\item (16.3.3) The minimal type of $ \mathsf{if} \ true \ \mathsf {then} \ false \ \mathsf {else} \  \{ \ \}$ is $Top$, the join of $Bool$ and $\{ \ \}$. However, it is hard to imagine that the programmer really intended to write this expression, after all, no operations can be performed on a value of type Top.
\end{enumerate}
\end{document}
