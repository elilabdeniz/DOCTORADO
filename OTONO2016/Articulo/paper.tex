\documentclass{article}

%%%% Packages
\usepackage{hyperref} %package for generating bookmarks
%\usepackage[utf8]{inputenc}
%\usepackage[T1]{fontenc}
\usepackage{graphicx}
%used for figure, for stopping auto float
\usepackage{float}
\usepackage{fancyhdr}
\usepackage{amsmath}
\usepackage{amssymb}
\usepackage{stackengine}

%Multiline comment
\usepackage{verbatim}
%for logic proofs
\usepackage{proof}
\usepackage{pdflscape}
% Package that contains captionof
\usepackage{caption}
% Package for code listings
\usepackage{listings}
\lstset{language=ML} 

%\usepackage{color}
\usepackage[usenames, dvipsnames]{color}
%\setcounter{secnumdepth}{6}
\usepackage{authblk}

\title{\bf Overloading}
\author{Elizabeth Labrada Deniz
\thanks{Funded by grant CONICYT, CONICYT-PCHA/Doctorado Nacional/2015-63140148}}
\affil{Computer Science Department (DCC), University of Chile, Chile}
\date{}
\setcounter{Maxaffil}{0}
\renewcommand\Affilfont{\itshape\small}

\begin{document}
	\maketitle
	\renewcommand{\abstractname}{Abstract}	
	\begin{abstract}
	\end{abstract}	
\section{Introduction}
Explicar Overloadin, ejemplos, para que sirve. Hablar de overl
\section{Concepts}\label{section:concepts}
\subsection{Explicit substitution}
\subsection{Local Type Inference}
\subsection{Polymorphism}
\section{Ad-hoc Polymorphism}\label{section:polymorphism}
\subsection{Type classes}
\subsection{Featherweight Java with dynamic and static overloading}
\end{document}
