\documentclass [proof]{article}

\usepackage{caption}
\usepackage{amsmath}
\usepackage{proof}
\usepackage[T1]{fontenc}
\usepackage{ebproof}
\usepackage{amssymb} % To provide the \varnothing symbol
\usepackage[latin1]{inputenc} % acentos sin codigo
\usepackage{verbatim} % comentarios
\usepackage[spanish]{babel}
\usepackage{mathtools}
\usepackage{setspace}
\usepackage{soul}%sombreado
\usepackage[pdftex]{color}
\usepackage{semantic}
\usepackage{float}
\usepackage{definitions}
\usepackage{array}
\usepackage{longtable}
\usepackage{xcolor,colortbl}
\usepackage{multirow}
\usepackage[letterpaper]{geometry}


\newcommand{\nvector}[2][a]{#1_{1},#1_{2},#1_{3},\cdots #1_{#2}}
\newcommand{\vect}{(x_1,x_2,\dots,x_n)}
\newcommand\gij[3]{\Gamma \vdash #1 \Rightarrow #2 : #3 }
\newcommand\rulename[1]{\mathrm{(#1)}}
\newcommand{\nothing}{\varnothing} % different from \emptyset
\newcommand{\tto}{\longrightarrow}
\newcommand{\env}{{\Gamma \vdash}}
\newcommand{\lambdax}{\lambda x}
\newcommand\myeq{\stackrel{\mathclap{\normalfont\mbox{def}}}{=}}
\begin{document}

%\setlength{\oddsidemargin}{-12mm}
%\setlength{\evensidemargin}{0mm}
%\setlength{\topmargin}{-20mm}
%\setlength{\textwidth}{125mm}
%\setlength{\textheight}{195mm}


\begin{comment}
$0^0$ es una expresi�n indefinida.
Si $a>0$ entonces $a^0=1$ pero $0^a=0.$
Sin embargo, convenir en que $0^0=1$ es adecuado para que
algunas f�rmulas se puedan expresar de manera sencilla,
sin recurrir a casos especiales, por ejemplo
$$e^x=\sum _{n=0}^{\infty }\frac{x^n}{n!}$$
$$(x+a)^n=\sum_{k=0}^n \binom{n}{k}x^k a^{n-k}$$
$\nvector[b]{m}$
$\nvector{n}$
$\vect$

$\infer{B} {A \& (A \rightarrow B)}$


$\infer[\mathrm{MP}]
{B}{A & \quad (A \rightarrow B)}$

$\infer[\rulename{TApp}] 
{\Gamma \vdash e_1 \: e_2 : \tau_2 }
{\Gamma \vdash e_1 : \tau_1 \rightarrow \tau_2 \quad 
\Gamma \vdash e_2: \tau_1}$


\[
\begin{tabular}{l l}
\infer[\rulename{CRefl}] 
{\tau \sim \tau}{} &
\infer[\rulename{CFun}] 
{\tau_1 \rightarrow \tau_2 \sim \tau'_1 \rightarrow \tau'_2}
{\tau_1 \rightarrow \tau'_1 \quad \tau_2 \rightarrow \tau'_2} \\
\\
\infer[\rulename{CUnR}] 
{\tau \sim  \: ?}{} &
\infer[\rulename{CUnL}] 
{ ? \sim \tau}{} 
\end{tabular}
\]
\end{comment}

%%%%%%%%%%%%%%%%%%%%%%%%%%%%%%%%%%%%%%%%%%%%%%%%reglas  separadas
\begin{comment}
$\infer[\rulename{EApp_1}]
{t_1 t_2 \tto t'_1 t_2}{(t_1 \tto t'_1)}$

$\infer[\rulename{EApp_2}]
{v_1 t_2  \tto v_1 t'_2} {t_2 t'_2}$

$\infer[\rulename{EAppAbs}]
{(\lambda x:T_{11}.t_{12}) \ v_2 \tto [x \mapsto v_2]t_{12} }{}
$

$\infer [\rulename {EProj}]
{t_1.l \tto t'_1.l}{t_1 \tto t'_1}$

$\infer [\rulename {ERcd}]
{ \{l_i = v_i ^ {i \in 1 \cdots j-1}, l_j = t_j, l_k = t_k ^ {k \in j+1 \cdots n}\} \\ 
\tto \{ l_i = v_i ^ {i \in 1 \cdots j-1}, l_j = t'_j, l_k = t_k ^ {k \in j+1 \cdots n}\} }{t_j \tto t'_j}$

$\infer [\rulename{ELetV}]
{let \ x=v_1 \ in \ t_2 \tto [x \mapsto v_1]t_2}{} 
$

$\infer [\rulename{ELet}]
{let \ x = t_1 \ in \ t_2 \tto let \ x \ = t'_1 \ in \ t_2}{t_1 \tto t'_1}
$

$\infer [\rulename {EFixBeta}]
{fix(\lambda x:T_1:t_2)\tto [x \mapsto (fix (\lambda x:T_1.t_2))]t_2}{}
$

$\infer[\rulename {EFix}]
{fix \ t_1 \tto fix \ t'_1}{t_1 \tto t'_1}
$

$\infer [\rulename{TVar}]
{\env \ x:T}{x:T \in \Gamma}
$

$\infer [\rulename{TAbs}]
{\env \lambda x : T_1.t_2:T_1 \to T_2}{\Gamma , x:T_1 \vdash t_2:T_2}
$

$\infer [\rulename{TApp}]
{\env t_1 t_2 : T_{12}}{\env t_1 : T_{11} \to T_{12} & \env t_2 : T_{11}}
$

$\infer [\rulename{TRcd}]
{{\env \{l_i=t_i^{i \in 1 \cdots n}\}:\{l_i:T_i^{i \in 1 \cdots n}\}}}{for \ each \ i \ \ \env t_i:T_i}
$

$\infer [\rulename {TProj}]
{\env t_1.l_j:T_j}{\env t_1:\{l_i:T_i^{i \in 1 \cdots n}\}}
$

$ \infer [\rulename {TLet}]
{\env \ let \ x = t_1 \ in \ t_2 : T_2}{\env t_1:T_1 & \Gamma, x:T_1 \vdash t_2:T_2}
$

$\infer [\rulename {TFix}]
{\env fix \ t_1:T_1}{\env t_1:T_1 \to T_1}
$

$letrec \ x:T_1 = t_1 \ in \ t_2 \myeq let \ x = fix(\lambda x:T_1.t1)\ in \ t_2$
\end{comment}
%%%%%%%%%%%%%%%%%%%%%%%%%%%%%%%%%%%%%%%%%%%%%%%%reglas  separadas

%%%%%%%%%%%%%%%%%%%%%%%%%%%%%%%%%%%%%%%%%%%%%%%%Sintaxis
\begin{comment}
\begin{table}[htbp]
\begin{center}
\begin{tabular}{l c l r}
$Syntax$ &&& \\ 
$t$ & $::=$ & &$terms:$\\ 
&&$x$&$variable$\\ 
&&$\lambda x:T.t$&$abstraction$\\
&&$t \ t$&$application$\\
&&$let \ x = t \ in \ t$&$let \ binding$\\
&&${l_i=t_i^{i \in 1 \cdots n}}$&$record$\\
&&$t.l$&$projection$\\
&&$fix \ t$&$fixed \ point\ of \ t$\\
&&&\\
$v$ & $::=$ & &$values:$\\
&&$\lambda x:T.t$&$abstraction \ value$\\
&&$\{l_i=v_i^{i \in 1 \cdots n}\}$&$record \ value$\\
&&&\\
$T$&$::=$&&$types:$\\
&&$T \to T$&$type \ of \ functions$\\
&&$\{l_i=T_i^{i \in 1 \cdots n}\}$&$type \ of \ record$\\
&&&\\
$\Gamma$&$::=$&&$contexts:$\\
&&$\varnothing$&$empty \ context:$\\
&&$\Gamma , x:T$&$term \ variable \ binding$\\
\end{tabular}
\end{center}
\end{table}
%%%%%%%%%%%%%%%%%%%%%%%%%%%%%%%%%%%%%%%%%%%%%%%%Sintaxis

%%%%%%%%%%%%%%%%%%%%%%%%%%%%%%%%%%%%%%%%%%%%%%%%Typing
\renewcommand{\arraystretch}{2}% 
\begin{table}[htbp]
\begin{center}
\begin{tabular}{l c r}
\multicolumn{2}{l}{$Typing$}&$\env t:T$\\
&$\infer []{\env \ x:T}{x:T \in \Gamma}$&$\rulename{TVar}$\\
&$\infer []{\env \lambda x : T_1.t_2:T_1 \to T_2}{\Gamma , x:T_1 \vdash t_2:T_2}$&$\rulename{TAbs}$\\
&$\infer []{\env t_1 t_2 : T_{12}}{\env t_1 : T_{11} \to T_{12} & \env t_2 : T_{11}}$&$\rulename{TApp}$\\
&$\infer []{{\env \{l_i=t_i^{i \in 1 \cdots n}\}:\{l_i:T_i^{i \in 1 \cdots n}\}}}{for \ each \ i \ \ \env t_i:T_i}$&$\rulename{TRcd}$\\
&$\infer []{\env t_1.l_j:T_j}{\env t_1:\{l_i:T_i^{i \in 1 \cdots n}\}}
$&$\rulename {TProj}$\\
&$ \infer []{\env \ let \ x = t_1 \ in \ t_2 : T_2}{\env t_1:T_1 & \Gamma, x:T_1 \vdash t_2:T_2}$&$\rulename {TLet}$\\
&$\infer []{\env fix \ t_1:T_1}{\env t_1:T_1 \to T_1}$&$\rulename {TFix}$\\
\end{tabular}
\end{center}
\end{table}
%%%%%%%%%%%%%%%%%%%%%%%%%%%%%%%%%%%%%%%%%%%%%%%%Typing

%%%%%%%%%%%%%%%%%%%%%%%%%%%%%%%%%%%%%%%%%%%%%%%%Evaluation
\renewcommand{\arraystretch}{2}% 
\begin{table}[htbp]
\begin{center}
\begin{tabular}{l c r}
\multicolumn{2}{l}{$Evaluation$}&$t \tto t'$\\
&$\infer[]{t_1 t_2 \tto t'_1 t_2}{(t_1 \tto t'_1)}$&$\rulename{EApp_1}$\\
&$\infer[]{v_1 t_2  \tto v_1 t'_2} {t_2 t'_2}$&$\rulename{EApp_2}$\\
&$\infer[]{(\lambda x:T_{11}.t_{12}) \ v_2 \tto [x \mapsto v_2]t_{12}}{}
$&$\rulename{EAppAbs}$\\
&$\infer []{t_1.l \tto t'_1.l}{t_1 \tto t'_1}$&$\rulename {EProj}$\\
&$\infer []
{ \{l_i = v_i ^ {i \in 1 \cdots j-1}, l_j = t_j, l_k = t_k ^ {k \in j+1 \cdots n}\} \tto \{ l_i = v_i ^ {i \in 1 \cdots j-1}, l_j = t'_j, l_k = t_k ^ {k \in j+1 \cdots n}\} }{t_j \tto t'_j}$&$\rulename {ERcd}$\\
&$\infer []{let \ x=v_1 \ in \ t_2 \tto [x \mapsto v_1]t_2}{}$&$\rulename{ELetV}$\\
&$\infer [] {let \ x = t_1 \ in \ t_2 \tto let \ x \ = t'_1 \ in \ t_2}{t_1 \tto t'_1}$&$\rulename{ELet}$\\
&$\infer []{fix(\lambda x:T_1:t_2)\tto [x \mapsto (fix (\lambda x:T_1.t_2))]t_2}{}$&$\rulename {EFixBeta}$\\
&$\infer[]{fix \ t_1 \tto fix \ t'_1}{t_1 \tto t'_1}$&$\rulename {EFix}$\\
&$letrec \ x:T_1 = t_1 \ in \ t_2 \myeq let \ x = fix(\lambda x:T_1.t1)\ in \ t_2$&\\
\end{tabular}
\end{center}
\end{table}

\begin{equation*} 
\begin{split} 
x &= 2 + 2\\ 
&= 2 + 2 + 3 - 3\\ 
&= 4. 
\end{split} 
\end{equation*}
%%%%%%%%%%%%%%%%%%%%%%%%%%%%%%%%%%%%%%%%%%%%%%%%Evaluation
\end{comment}
\newgeometry{left=1cm,bottom=0.1cm}


\begin{table}[htbp]
\setlength{\tabcolsep}{2pt}
\setlength{\extrarowheight}{20pt}
%\setlength{\extracolumnheight}{20pt}
%\renewcommand{\arraystretch}{2}% 
\begin{center}
\begin{tabular}{|l|l|}
\hline

\begin{tabular} {l}
\setlength{\extrarowheight}{0pt}
\begin{tabular}{l c l r}
$Syntax:$ &&& \\ 
$t$ & $::=$ & &$terms:$\\ 
&&$x$&$variable$\\ 
&&$\lambda x:T.t$&$abstraction$\\
&&$t \ t$&$application$\\
&&$let \ x = t \ in \ t$&$let \ binding$\\
&&$\{l_i=t_i^{i \in 1 \cdots n}\}$&$record$\\
&&$t.l$&$projection$\\
&&$fix \ t$&$fixed \ point\ of \ t$\\
&&&\\
$v$ & $::=$ & &$values:$\\
&&$\lambda x:T.t$&$abstraction \ value$\\
&&$\{l_i=v_i^{i \in 1 \cdots n}\}$&$record \ value$\\
&&&\\
$T$&$::=$&&$types:$\\
&&$T \to T$&$type \ of \ functions$\\
&&$\{l_i=T_i^{i \in 1 \cdots n}\}$&$type \ of \ record$\\
&&&\\
$\Gamma$&$::=$&&$contexts:$\\
&&$\varnothing$&$empty \ context:$\\
&&$\Gamma , x:T$&$term \ variable \ binding$\\
\end{tabular}\\
\begin{tabular}{l c r}
\multicolumn{2}{l}{$Typing:$}&$\env t:T$\\
&$\inference {x:T \in \Gamma}{\env \ x:T}$&$\rulename{TVar}$\\
&$\inference {\Gamma , x:T_1 \vdash t_2:T_2}{\env \lambda x : T_1.t_2:T_1 \to T_2}$&$\rulename{TAbs}$\\
&$\inference {\env t_1 : T_{11} \to T_{12} & \env t_2 : T_{11}}{\env t_1 t_2 : T_{12}}$&$\rulename{TApp}$\\
&$\inference {for \ each \ i \ \ \env t_i:T_i}{{\env \{l_i=t_i^{i \in 1 \cdots n}\}:\{l_i:T_i^{i \in 1 \cdots n}\}}}$&$\rulename{TRcd}$\\
&$\inference {\env t_1:\{l_i:T_i^{i \in 1 \cdots n}\}}{\env t_1.l_j:T_j}
$&$\rulename {TProj}$\\
&$\inference {\env t_1:T_1 \to T_1}{\env fix \ t_1:T_1}$&$\rulename {TFix}$\\
\end{tabular}\\
\end{tabular}
&
\begin{tabular}{l c r}
&$ \inference {\env t_1:T_1 & \Gamma, x:T_1 \vdash t_2:T_2}{\env \ let \ x = t_1 \ in \ t_2 : T_2}$&$\rulename {TLet}$\\
\multicolumn{2}{l}{$Evaluation:$}&$t \tto t'$\\
&$\inference{t_1 \tto t'_1}{t_1 t_2 \tto t'_1 t_2}$&$\rulename{EApp_1}$\\
&$\inference {t_2 \tto t'_2}{v_1 t_2  \tto v_1 t'_2}$&$\rulename{EApp_2}$\\
&${(\lambda x:T_{11}.t_{12}) \ v_2 \tto [x \mapsto v_2]t_{12}}$&$\rulename{EAppAbs}$\\
&$\{ l_i = v_i ^ { i \in 1 \cdots n} \}.l_j \tto v_j$&$\rulename{EProjRcd}$\\
&$\inference {t_1 \tto t'_1}{t_1.l \tto t'_1.l}$&$\rulename {EProj}$\\
&\setlength{\extrarowheight}{0pt} $\inference {t_j \tto t'_j}{ \vspace{-0.6cm}\begin{block}  \{l_i = v_i ^ {i \in 1 \cdots j-1}, l_j = t_j, l_k = t_k ^ {k \in j+1 \cdots n}\} \\
\tto \{ l_i = v_i ^ {i \in 1 \cdots j-1}, l_j = t'_j, l_k = t_k ^ {k \in j+1 \cdots n}\} \end{block}}$&$\rulename {ERcd}$\\
&${let \ x=v_1 \ in \ t_2 \tto [x \mapsto v_1]t_2}$&$\rulename{ELetV}$\\
&$\inference {t_1 \tto t'_1}{let \ x = t_1 \ in \ t_2 \tto let \ x \ = t'_1 \ in \ t_2}$&$\rulename{ELet}$\\
&${fix(\lambda x:T_1.t_2)\tto [x \mapsto (fix (\lambda x:T_1.t_2))]t_2}$&$\rulename {EFixBeta}$\\
&$\inference{t_1 \tto t'_1}{fix \ t_1 \tto fix \ t'_1}$&$\rulename {EFix}$\\
\multicolumn{2}{l}{$Derived \ Forms:$}&\\
&$letrec \ x:T_1 = t_1 \ in \ t_2 \myeq let \ x = fix(\lambda x:T_1.t1)\ in \ t_2$&\\
\end{tabular}\\ 
\hline
\end{tabular}

\captionsetup{justification=centering}
\caption{Simply typed lambda-calculus with Let binding, Records and General recursion.}
\label{tabla:sencilla}
\end{center}
\end{table}
\restoregeometry
%\setlength{\oddsidemargin}{0mm}
%\setlength{\evensidemargin}{0mm}
%\setlength{\topmargin}{0mm}
%\setlength{\marginparwidth}{0mm}
%\setlength{\marginparsep}{0mm}
%\setlength{\textheight}{195mm}

LEMMA [Inversion of the Typing Relation]:
\begin{enumerate}
\item If $\env \ x : R$ , then $x : R \in \Gamma$
\item If $\env \ \lambda x: T_1. t_2 : R$, then $R = T_1 \to R_2$ for some $R_2$, with $\Gamma$, $x : T_1\vdash t_2 : R_2$.
\item If $\env \  t_1 \ t_2 : R$, then there is some type $T_{11}$ such that $\env t_1 : T_{11} \to R$ and $\env t_2 : T_{11}$.
\item If $\env \ let \ x = t_1 \ in \ t_2 : R$, then  there is some type $T_{1}$ such that $\env t_1 : T_{1}$, with $\Gamma$, $x : T_1\vdash t_2 : R$.
\item If $\env \ \{l_i = t_i ^ {\ i \in 1 \cdots n}\} :R$, then there are some types $R_i ^ {\ i \in 1\cdots n}$ such that for each $i$ is satisfied that $\env t_i:R_i$ and $R = \{l_i=R_i^{\ i \in 1 \cdots n}\}$.
\item If $\env \ t.l_j : R$, then there is some type $\{l_i=R_i^{\ i \in 1 \cdots n}\}$ such that $\env t : \{l_i=R_i^{\ i \in 1 \cdots n}\}$ and $R = R_j$.
\item $\env \ fix \ t_1 : R$ then $\env \ t1 : R \to R$.
\end{enumerate}
$Proof:$ Immediate from the definition of the typing relation.\ \\

LEMMA [Canonical Forms]:
\begin{enumerate}
\item If $v$ is a value of type $T_1 \to T_2$, then $v = \lambda x : T_1. t_2$.
\item If $v$ is a value of type $\{l_i=T_i^{\ i \in 1 \cdots n}\}$ then $v = \{l_i=v_i^{\ i \in 1 \cdots n}\}$, and $\env v_i:T_i {\ i \in 1 \cdots n}$.
\end{enumerate}
$Proof:$ Straightforward.\ \\

THEOREM [Progress]: Suppose $t$ is a closed, well-typed term (that is, $ \vdash \ t' : T$ for some $T$). Then either $t$ is a value or else there is some $t'$ with $t \tto t'$.
Proof: Straightforward induction on typing derivations.
\begin{enumerate}
\item The variable case cannot occur (because $t$ is closed).
\item The abstraction case is immediate, since abstractions are values.
\item For application, where $t = t_1 \ t_2$, with $\vdash \ t_1 : T_{11} \tto T_{12}$ and $\vdash \ t_2 : T_{11}$, for the inversion lemma. Then, by the induction hypothesis, either $t_1$ is a value or else it can make a step of evaluation, and likewise $t_2$. If $t_1$ can take a step, then rule $EApp1$ applies to $t$. If $t_1$ is a value and $t_2$ can take a step, then rule $EApp2$ applies. Finally, if both $t_1$ and $t_2$ are values, then the canonical forms lemma tells us that $t_1$ has the form $\lambda x: T_{11}.t_{12}$, and so rule $EAppAbs$ applies to $t$.
\item If $t = (let \ x = t_1 \ in \ t_2) $, then $\vdash \ t_1 : T_{1}$ and $\vdash \ t_2 : T_{2}$, for the inversion lemma. Then, by the induction hypothesis, either $t_1$ is a value or else it can make a step of evaluation. If $t_1$ can take a step, then rule $ELet$ applies to $t$. If $t_1$ is a value, then rule $ELetV$ applies.
\item If $t = \{l_i=t_i^{\ i \in 1 \cdots n}\}$, then there are some types $T_i ^ {\ i \in 1\cdots n}$ such that for each $i$ is satisfied that $\env t_i:T_i$ and $T = \{l_i=T_i^{\ i \in 1 \cdots n}\}$. By the induction hypothesis, for each $t_i^{\ i \in 1 \cdots n}$, either it is a value or else it can make a step of evaluation. If all the $t_i^{\ i \in 1\cdots n}$ are values, then $t$  is a value. If all the $t_i^{\ i \in 1\cdots n}$ are not values, then there is $t_j$ such that it can take a step, then rule $ERcd$ applies to $t$. 
\item If $t = t'.l_j$, then there is some type $\{l_i=T_i^{\ i \in 1 \cdots n}\}$ such that $\env t' : \{l_i=T_i^{\ i \in 1 \cdots n}\}$ and $T = T_j$. By the induction hypothesis, either $t'$ is a value or else it can make a step of evaluation. If $t'$ can take a step, then rule $EProj$ applies to $t$. If $t'$ is a value, then rule $EProjRcd$ applies.
\item If $t = fix \ t'$, then $\env \ t' : T \to T$, for the inversion lemma. By the induction hypothesis, either $t'$ is a value or else it can make a step of evaluation. If $t'$ can take a step, then rule $EFix$ applies to $t$. If $t'$ is a value, then the canonical forms lemma tells us that $t'$ has the form $\lambda x: T.t_1$, and so rule $EFixBeta$ applies to $t$.
\end{enumerate}

LEMMA[Permutation]: If $\env \ t : T$ and $\Delta$ is a permutation of $\Gamma$, then $\Delta \vdash \ t : T$.\\
$Proof$: Straightforward induction on typing derivations.\ \\

LEMMA[Weakening]: If $\env \ t : T$ and $x \not \in dom(\Gamma)$, then $\Gamma, x : S \vdash \ t:T$.\\
$Proof$: Straightforward induction on typing derivations.\ \\

LEMMA [Preservation of types under substitution]: If $\Gamma , x:S \ \vdash \ t : T$ and $\env s : S$, then $\env \ [x \mapsto s]t : T$.\\
$Proof$: By induction on a derivation of the statement $ \Gamma, x:S \ \vdash \ t : T$. For a given derivation, we proceed by cases on the final typing rule used in the proof.
\begin{enumerate}
\item $Case \ TVar: t = z$ with $z:T \in (\Gamma , x:S)$. There are two sub-cases to consider, depending on whether $z$ is $x$ or another variable. If $z = x$, then $[x \mapsto s]z = s$. The required result is then $\env \ s : S$, which is among the assumptions of the lemma. Otherwise, $[x \mapsto s]z = z$, and the desired result is immediate.
\item $Case \ TAbs$: $t = \lambda y:T_2.t_1, with \ T = T_2 \to T_1$ and $ \Gamma, x:S, y:T_2 \ \vdash \ t_1 : T_1$. By convention, we may assume $x \not=  y$ and $y \not\in FV(s)$. Using permutation on the given subderivation, we obtain $\Gamma, y:T_2, x:S \ \vdash \ t_1 : T_1$. Using weakening on the other given derivation $(\env \ s : S)$, we obtain $\Gamma, y:T_2 \ \vdash \ s : S$. Now, by the induction hypothesis, $\Gamma , y:T_2 \ \vdash \ [x \mapsto s]t_1 : T_1$. By $TAbs, \env \ \lambda y: T_2.[x \mapsto s]t_1 : T_2 \to T_1$. But this is precisely the needed result,since, by the definition of substitution, $[x \mapsto s]t = \lambda y:T_1. [x \mapsto s]t_1$.
\item $Case \ TApp$: $t = t_1 \ t_2, \Gamma, x:S\ \vdash \ t_1 : T_2 \to T_1, \Gamma, x:S\ \vdash \ t_2 : T_2, T = T_1$.
By the induction hypothesis, $\env \ [x \mapsto s]t_1: T_2 \to T_1$ and $\env \ [x \mapsto s]t_2 : T_2.$ By $TApp$, $\env \ [x \mapsto s]t_1 \ [x \mapsto s]t_2:T$, then $\env \ [x \mapsto s] (t_1 \ t_2):T$.
\item $Case \ TRcd$: $t = \{l_i = t_i^ {\ i \in 1 \cdots n}\}$, for each $i, \Gamma, x:S \ \vdash \ t_i :T_i$, $T = \{l_i = T_i^ {\ i \in 1 \cdots n}\}$.
By the induction hypothesis, for each $i, \env \ [x \mapsto s]t_i :T_i$. By the $TRcd$, $\env \{l_i = [x \mapsto s]t_i^ {\ i \in 1 \cdots n}\} : T$, then $\env [x \mapsto s]\{l_i = t_i^ {\ i \in 1 \cdots n}\} : T$.
\item $Case \ TLet$: $t = (let \ y = t_1 \ in \ t_2), \Gamma, x:S \ \vdash \ t_1 :T_1, \Gamma, x:S, \ y:T_1 \ \vdash \ t_2 :T_2, \ T = T_2$. By convention, we may assume $x \not=  y$ and $y \not\in FV(s)$. Using permutation on the given subderivation, we obtain $\Gamma, y:T_1, x:S \ \vdash \ t_2 : T_2$. Using weakening on the other given derivation $(\env \ s : S)$, we obtain $\Gamma, y:T_1 \ \vdash \ s : S$. Now, by the induction hypothesis, $\Gamma , y:T_1 \ \vdash \ [x \mapsto s]t_2 : T_2$ and $\Gamma \ \vdash \ [x \mapsto s]t_1 : T_1$. By $TLet, \env \ let \ y: [x \mapsto s]t_1\ in \ [x \mapsto s]t_2 : T_2$. But this is precisely the needed result,since, $[x \mapsto s]t = (let \ y = [x \mapsto s]t_1\ in \ [x \mapsto s]t_2)$.
\item $Case \ TProj$: $t = t_1.l_j, \ \Gamma, x : S \vdash \ t_1 : \{l_i = T_i^{\ i \in 1 \cdots n}\}$ and $T = T_j$. By the induction hypothesis, $\env \ [x \mapsto s]t_1 : \{l_i = T_i^{\ i \in 1 \cdots n}\}$. By $TProj$, $\env \ [x \mapsto s]t_1.l_j : T$. But this is precisely the needed result,since, $[x \mapsto s]t =  [x \mapsto s]t_1.l_j$.
\item $Case \ TFix$: $t = fix \ t_1$ and $\Gamma, x:S \vdash t_1 : T \to T$. By the induction hypothesis, $\env \ [x \mapsto s]t_1 : T \to T$. By $TFix$, $\env \ fix \ [x \mapsto s]t_1 : T$. But this is precisely the needed result,since, $[x \mapsto s]t = fix \ [x \mapsto s]t_1$.
\end{enumerate}

LEMMA[Preservation]: If $\env \ t : T$ and $t \tto t'$, then $\env \ t' : T$.\\
$Proof$: By induction on a derivation of $t : T$. At each step of the induction, we assume that the desired property holds for all subderivations (i.e., that if $s : S$ and $s \tto s'$, then $s' : S$, whenever s : S is proved by a subderivation of the present one) and proceed by case analysis on the final rule in the derivation.
\begin{enumerate}
\item $Case \ TVar$: It cannot be the case that $t \tto t'$ for any $t'$, and the requirements of the theorem are vacuously satisfied.
\item $Case \ TAbs$: $t = \lambda x:T_2.t_1, with \ T = T_2 \to T_1$ and $ \Gamma, x:T_2 \ \vdash \ t_1 : T_1$. If the last rule in the derivation is $TAbs$, then we know from the form of this rule that t must be a function $\lambda x:T_2.t_1$ and $T$ must be $T_2 \to T_1$, with $ \Gamma, x:T_2 \ \vdash \ t_1 : T_1$. But then $t$ is a value, so it cannot be the case that $t \tto t'$ for any $t'$, and the requirements of the theorem are vacuously satisfied.
\item $Case \ TApp$: $t = t_1 \ t_2, \ \vdash \ t_1 : T_2 \to T$ and $\vdash \ t_2 : T_2$. If the last rule in the derivation is $TApp$, then we know from the form of this rule that $t$ must have the form $t_1 \ t_2$, for some $t_1$ and $t_2$. We must also have a subderivation with conclusions $\vdash t_1 : T_2 \to T$ and $\vdash t_2 : T_2$. Now, looking at the evaluation rules with this form on the left-hand side, we find that there are three rules by which $t \tto t'$ can be derived: $EApp1$, $EApp2$ and $EAppAbs$. We consider each case separately.
\begin{itemize}
\item $Subcase \ EApp1$: $t_1 \tto t_1', \ t' = t_1' \ t_2$. By the induction hypothesis, $\vdash t_1' : T_2 \to T$, then we can apply rule $TApp$, to conclude that $\vdash t_1' \ t_2: T$ , that is $\vdash t' : T$.
\item $Subcase \ EApp2$: $t_2 \tto t_2', \ t' = t_1 \ t_2'$. By the induction hypothesis, $\vdash t_2' : T_2 $, then we can apply rule $TApp$, to conclude that $\vdash t_1 \ t_2' : T$, that is $\vdash t' : T$.
\item $Subcase \ EAppAbs$: $t_1 = \lambda x:T_{1}.t_{12}$, $t_2 = v_2$, $t' = [x \mapsto v_2]t_{12}$ and $\vdash t_{12}: T$ for the inversion lemma. The resulting term $\vdash [x \mapsto v_2]t_{12}:T$, for the Substitution lemma, that is $\vdash t' : T$.
\end{itemize}
\item $Case \ TRcd$: $t = \{l_i = t_i^ {\ i \in 1 \cdots n}\}$, for each $i, \vdash \ t_i :T_i$, $T = \{l_i = T_i^ {\ i \in 1 \cdots n}\}$.
\begin{itemize}
\item If for each $i$, $t_i = v_i$, then $t$ is a value, so it cannot be the case that $t \tto t'$ for any $t'$, and the requirements of the theorem are vacuously satisfied.
\item $Subcase \ ERcd$: $t = \{l_i = v_i^ {\ i \in 1 \cdots j-1}, l_j = t_j, l_k = t_k^ {\ k \in j+1 \cdots n}\}$, $t_j \tto t_j'$, $t' = \{l_i = v_i^ {\ i \in 1 \cdots j-1}, l_j = t_j', l_k = t_k^ {\ k \in j+1 \cdots n}\}$ and $\vdash t_j : T_j$. By the induction hypothesis, $\vdash t_j' : T_j$, then we can apply rule $TRcd$, to conclude that $\vdash t' : T$.
\end{itemize}
\item $Case \ TLet$: $t = (let \ x = t_1 \ in \ t_2), \Gamma \vdash \ t_1 :T_1$ and $\Gamma, x:T_1 \ \vdash \ t_2 :T$. If the last rule in the derivation is $TLet$, then we know from the form of this rule that $t$ must have the form $(let \ x = t_1 \ in \ t_2)$, for some $t_1$ and $t_2$. We must also have a subderivation with conclusions $\vdash t_1 : T_1$ and $\vdash t_2 : T_2$. Now, looking at the evaluation rules with $Let$ form on the left-hand side, we find that there are two rules by which $t \tto t'$ can be derived: $ELetV$ and $ELet$. We consider each case separately.
\begin{itemize}
\item $Subcase \ ELetV$: $t_1 = v_1$, $t' = [x \mapsto v_1]t_2$, and $\vdash t_2 : T$. Then for the substitution lemma $t': T$.
\item $Subcase \ ELet$: $t_1 \tto t_1'$ and $t'= (let \ x = t_1' \ in \ t_2)$. By the induction hypothesis, $\vdash t_1' : T_1 $, then we can apply rule $TLet$, to conclude that $\vdash (let \ x = t_1' \ in \ t_2) : T$, that is $\vdash t' : T$.
\end{itemize}
\item $Case \ TProj$: $t = t_1.l_j,\ \vdash \ t_1 : \{l_i = T_i^{\ i \in 1 \cdots n}\}$ and $T = T_j$. If the last rule in the derivation is $TProj$, then we know from the form of this rule that $t$ must have the form $t_1.l_j$. We must also have a subderivation with conclusions $\vdash t_1 : \{l_i = T_i^{\ i \in 1 \cdots n}\}$ and $T = T_j$. Now, looking at the evaluation rules with this form on the left-hand side, we find that there are two rules by which $t \tto t'$ can be derived: $EProjRcd$ and $EProj$. We consider each case separately.
\begin{itemize}
\item $Subcase \ EProjRcd$: $t_1 = \{l_i = v_i^{\ i \in 1 \cdots n}\}$ and $t' = v_j$.This means we are finished, since we know $\vdash v_j : T_j$ and $T =T_j$.
\item $Subcase \ EProj$: $t_1 \tto t_1'$ and $t' = t_1'.t_j$. By the induction hypothesis, $\vdash t_1' : \{l_i = T_i^{\ i \in 1 \cdots n}\} $, then we can apply rule $TLet$, to conclude that $\vdash t_1'.t_j : T_j$, that is $\vdash t' : T$.
\end{itemize}
\item $Case \ TFix$: $t = fix \ t_1$ and $\vdash t_1 : T \to T$. If the last rule in the derivation is $TFix$, then we know from the form of this rule that $t$ must have the form $fix \ t_1$, for some $t_1$. We must also have a subderivation with conclusions $\vdash t_1 : T \to T$. Now, looking at the evaluation rules with $fix$ on the left-hand side, we find that there are two rules by which $t \tto t'$ can be derived: $EFixBeta$ and $EFix$. We consider each case separately.
\begin{itemize}
\item $Subcase \ EFixBeta$: $t_1 = \lambda x : T_1.t_2, \ t' = [x \mapsto (fix \ (\lambda x : T_1.t_2))]t_2$, $\vdash t_1 : T \to T$ and $\vdash t_2 : T$, for the inversion lemma. Then, by the substitution lemma, $ \vdash [x \mapsto (fix \ (\lambda x : T_1.t_2))]t_2 : T$, which is what we need.
\item $Subcase \ EFix$: $t_1 \tto t_1'$ and $t' = fix \ t_1'$. By the induction hypothesis, $\vdash t_1' : T \to T $, then we can apply rule $TFix$, to conclude that $\vdash fix \ t_1' : T$, that is $\vdash t' : T$.          
\end{itemize}
\end{enumerate}
\end{document}
