%\documentclass
%\documentclass[preprint,authoryear,sort,9pt,twocolumn,nocopyrightspace]
%\documentclass[preprint,authoryear,sort&compress,9pt,nocopyrightspace]{sigplanconf}
\documentclass[preprint,authoryear,sort&compress,9pt,nocopyrightspace]{article}


%\usepackage{changepage}
%\usepackage{lipsum}
%\usepackage{caption}
%\usepackage{proof}
%\usepackage[T1]{fontenc}
%\usepackage{ebproof}
%\usepackage[latin1]{inputenc} % acentos sin codigo
%\usepackage[spanish]{babel}
%\usepackage{setspace}
%\usepackage{soul}%sombreado
%\usepackage[pdftex]{color}
%\usepackage{float}
%\usepackage{array}
%\usepackage{longtable}
%\usepackage{xcolor,colortbl}
%\usepackage{semantic}
%\usepackage{definitions}
%\usepackage{multirow}
%\usepackage[letterpaper]{geometry}
\usepackage{mathtools}
\usepackage{verbatim} % comentarios
%\usepackage{sansmath}
%\usepackage{amssymb} % To provide the \varnothing symbol
%\usepackage{amsmath}



\newcommand{\nvector}[2][a]{#1_{1},#1_{2},#1_{3},\cdots #1_{#2}}
\newcommand{\vect}{(x_1,x_2,\dots,x_n)}
\newcommand\gij[3]{\Gamma \vdash #1 \Rightarrow #2 : #3 }


\newcommand\rulename[1]{\mathrm{(#1)}}
\newcommand{\nothing}{\varnothing} % different from \emptyset
\newcommand{\tto}{\longrightarrow}
\newcommand{\lto}{\leftarrow}
\newcommand{\lambdax}{\lambda x}



\newcommand{\conf}[2][s]{(#2)[#1]}
\newcommand{\confxu}[1]{#1 [x,(v:T_1):s]}
\newcommand{\confxD}[1]{#1 [x,\{(\overline{v:T})\}:s]}
\newcommand{\confx}[1]{#1 [x,\{(\overline{v:T_1})\}:s]}
\newcommand{\confy}[1]{#1 [y, \{(\overline{v:T_1})\}:s]}
\newcommand{\confxE}[1]{#1 [x,(v:T_1):s']}
\newcommand{\confyE}[1]{#1 [y,(v:T_1):s']}
\newcommand{\subxD}{[x,(v:T_1):s]}
\newcommand{\subx}{[x,(v:T_1):s']}
\newcommand{\suby}{[y,(v:T_1):s']}
\newcommand{\SubxD}{x,(v:T_1):s}
\newcommand{\Subx}{x,(v:T_1):s'}
\newcommand{\Suby}{y,(v:T_1):s'}

\newcommand{\env}{{\emt,\Gamma \vdash \ }}
\newcommand{\envE}{{\emt, \Gamma , x:T_1 \vdash \ }}
\newcommand{\tyC}{{\Gamma \vdash_c \ }}
\newcommand{\enve}{{\emt  , x \conc T_1, \Gamma\vdash \ }}
\newcommand{\envC}{{\emt(s), \Gamma \vdash \ }}
\newcommand{\envEC}{{\emt(s), \Gamma , x:T_1 \vdash \ }}
\newcommand{\enveC}{{\emt(s), x \conc T_1, \Gamma \vdash \ }}


\newcommand{\ascrip}[1]{#1::T}
\newcommand{\ascripP}[2]{#1::#2}
\newcommand{\oletD}{\mathsf{olet} \ x : T = t \ \mathsf{in}  \ t}
\newcommand{\olet}{\mathsf{olet} \ x : T_1 = t_1 \ \mathsf{in}  \ t_2}
\newcommand{\oletP}[3]{\mathsf{olet} \ x : #1 = #2 \ \mathsf{in}  \ #3}
\newcommand{\app}[2]{#1 \ #2}
\newcommand{\appD}{t_1 \ t_2}
\newcommand{\abs}[3]{\lambda #1:#2. \ #3}
\newcommand{\absD}{\lambda x:T_1. \ t_2}

\newcommand{\truet}{\mathsf{true}}
\newcommand{\falset}{\mathsf{false}}
\newcommand{\boolt}{\mathsf{Bool}}
\newcommand{\intt}{\mathsf{Int}}
\newcommand{\nvt}{\mathsf{nv}}

\newcommand{\mtD}{T^{*}}
\newcommand{\mtP}[1]{#1^{*}}
\newcommand{\mtC}[1]{\{\overline {#1}\}}
\newcommand{\mtCu}[1]{\{ #1 \}}

\newcommand{\emt}{\varphi}
\newcommand{\conc}{:^{*}}
\newcommand\inferir{\stackrel{\mathclap{\normalfont\mbox{$\to$}}}{\in}}
\newcommand\chequear{\stackrel{\mathclap{\normalfont\mbox{$\lto$}}}{\in}}
\newcommand\myeq{\stackrel{\mathclap{\normalfont\mbox{def}}}{=}}
\providecommand{\norm}[1]{\lVert#1\rVert}
\usepackage{definitions}

%\usepackage{anyfontsize} % kill warnings about font sizes 
\usepackage{amsmath}     % Mathematics FTW!
\usepackage{amsthm}      % Warning, I've only proven it correct :).
\usepackage{amssymb}     % needed for \mathbb and mathcal
\usepackage{amsbsy}      % for \boldsymbol
\usepackage{semantic}    % for PLT
\usepackage{stmaryrd}   % provides some PLy symbols, like Scott Brackets
\usepackage{mathpartir}  % for \mathpar
\usepackage{color}
\usepackage{braket}      % for \set and \braket
\usepackage{mathtools}   % for \bigtimes
\usepackage{thmtools}    % to duplicate theorems in the appendix
\usepackage{marvosym} % for deniarius (gradual label)
\usepackage{wasysym}     % for \smiley
\usepackage{balance}     % for the references
\usepackage{xspace}
\usepackage{mathrsfs} % for mathscr font
\usepackage{wasysym}
\usepackage{proof}
\usepackage{yhmath}
\usepackage{rotating}
\usepackage{tikz-cd}
\usepackage{xcolor}
\usepackage{changepage}
\usepackage{centernot}
\usepackage{comment}
\usepackage{listings}
\usepackage{mdframed}
\usepackage{centernot}
\usepackage{mathtools}
\usepackage{xparse}
\usepackage{ upgreek }

\def\presuper#1#2%
  {\mathop{}%
   \mathopen{\vphantom{#2}}^{#1}%
   \kern-\scriptspace%
   #2}

\ExplSyntaxOn
\DeclareExpandableDocumentCommand{\IfNoValueOrEmptyTF}{mmm}
 {
  \IfNoValueTF{#1}{#2}
   {
    \tl_if_empty:nTF {#1} {#2} {#3}
   }
 }
\ExplSyntaxOff

\makeatletter
\newcommand{\xMapsto}[2][]{\ext@arrow 0599{\Mapstofill@}{#1}{#2}}
\def\Mapstofill@{\arrowfill@{\Mapstochar\Relbar}\Relbar\Rightarrow}
\makeatother

% Comment form:  requires amssymb and color
\newcommand{\mynote}[2]
    {{\color{red} \fbox{\bfseries\sffamily\scriptsize#1}
    {\small$\blacktriangleright$\textsf{\emph{#2}}$\blacktriangleleft$}}~}

%\definecolor{clxccolor}{HTML}{3F7D31}
%\definecolor{staticcolor}{HTML}{3F7D31}
\definecolor{clxccolor}{HTML}{009773}
\definecolor{staticcolor}{HTML}{009773}
\definecolor{propcolor}{HTML}{3F7D31}
\definecolor{mypurple}{HTML}{5B069D}
\definecolor{dynamic}{HTML}{D55E00}
\definecolor{memory}{HTML}{0072B2}
% \definecolor{clxccolor}{RGB}{182,109,255}
% \definecolor{staticcolor}{RGB}{182,109,255}
% \definecolor{dynamic}{RGB}{219,209,0}
%\definecolor{orange}{RGB}{0,146,146}
\newcommand{\myproposal}[2]
    {{\color{propcolor} \fbox{\bfseries\sffamily\scriptsize#1}
    {\small$\blacktriangleright$\textsf{\emph{#2}}$\blacktriangleleft$}}~}

% \renewcommand{\mynote}[2]{}
\newcommand{\et}[1]{\mynote{ET}{#1}}
\newcommand{\ron}[1]{\mynote{RG}{#1}}
\newcommand{\mt}[1]{\mynote{MT}{#1}}
\newcommand{\nl}[1]{\mynote{NL}{#1}}

\newcommand{\mtp}[1]{\myproposal{MT}{#1}}

%% block - used tog indent code or create newlines of math stuff.
\newenvironment{block}[1][t]
  {\begin{array}[#1]{@{}l@{}}}
  {\end{array}}

% Grey Box
\definecolor{lightgray}{gray}{0.90}
\newcommand{\Gbox}[1]{\colorbox{lightgray}{$#1$}}

\newcommand{\gbox}[1]{{\setlength{\fboxsep}{1pt}\colorbox{lightgray}{\tiny$#1$}}}


%
% Mathy definitions
% 

\declaretheorem{theorem}
\declaretheorem[sibling=theorem]{lemma}
\declaretheorem[sibling=theorem]{proposition}
\declaretheorem[sibling=theorem]{corollary}
\declaretheorem{definition}
\declaretheorem[style=remark,numbered=no]{case}
\declaretheorem[style=remark,numbered=no]{notation}
% \newtheorem{theorem}{Theorem}
% \newtheorem{lemma}[theorem]{Lemma}
% \newtheorem{proposition}[theorem]{Proposition}
% \newtheorem{corollary}[theorem]{Corollary}
%\newtheorem{definition}{Definition}
%\newtheorem{property}{Property}

%\theoremstyle{remark}
%\newtheorem*{case}{Case}
%\newtheorem*{notation}{Notation}


%
% PLy definitions
%
\newcommand{\sbrack}[1]{|[ #1 |]} % Scott brackets
\newcommand{\Sbrack}[1]{\left[\!\!\left[ #1 \right]\!\!\right]} % Scott brackets

\mathlig{[]}{\square}
\mathlig{|-s}{\vdash^{\!\forall}}
\mathlig{|-}{\vdash}
\mathlig{|}{\mid} % For BNFs
\mathlig{@>}{\leadsto} % for type-directed translations


\reservestyle{\oblang}{\mathsf}
\oblang{if[if\;],then[\;then\;],else[\;else\;],let[let\;],in[\;in\;]}

%
% Cleaned up macros
%

% syntactic sets
\newcommand{\Z}{\ensuremath{\mathbb{Z}}} % the set of integers
\newcommand{\Pow}{\ensuremath{\mathcal{P}}} % The powerset
% Use sans-serif for object language constructors
\newcommand{\Int}{\mathsf{Int}}
\newcommand{\String}{\mathsf{String}}
\newcommand{\Float}{\mathsf{Float}}
\newcommand{\Bool}{\mathsf{Bool}}
% Use smallcaps for the names of object language sets.
\newcommand{\oblset}[1]{\textsc{#1}}
\newcommand{\Label}{\oblset{Label}}
\newcommand{\Type}{\oblset{Type}}
\newcommand{\Var}{\oblset{Var}}
\newcommand{\Value}{\oblset{Value}}
\newcommand{\Term}{\oblset{Term}}
\newcommand{\Const}{\oblset{Const}}
\newcommand{\TTerm}{\oblset{TTerm}}
\newcommand{\XTerm}{\oblset{XTerm}}
\newcommand{\GTerm}{\oblset{GTerm}}
\newcommand{\SType}{\oblset{SType}}
\newcommand{\GType}{\oblset{GType}}
\newcommand{\EType}{\oblset{EType}}
\newcommand{\GEType}{\oblset{GEType}}
\newcommand{\GSType}{\oblset{GSType}}
\newcommand{\GLabel}{\oblset{GLabel}}
\newcommand{\setof}[1]{\set{\overline{#1}}}

% language
\newcommand{\fff}{\textsf{false}}
\newcommand{\ff}{\fff} % temporary - backward compat
\newcommand{\ttt}{\textsf{true}}
% \newcommand{\ite}[3]{\texttt{if}\;#1\;\texttt{then}\;#2\;\texttt{else}\;#3}
\newcommand{\ite}[3]{\<if> #1 \<then> #2 \<else> #3}
\newcommand{\iite}[4]{\<if>^{{\color{staticcolor} #1}} #2 \<then> #3 \<else> #4}
\newcommand{\lsec}{$\lambda_\text{SEC}$\xspace}
\newcommand{\lgsec}{$\lambda_{\consistent{\text{SEC}}}$\xspace}

% relations
\newcommand{\dom}{\mathit{dom}}
\newcommand{\cod}{\mathit{cod}}
\newcommand{\cmb}{\mathit{equate}}
\newcommand{\lvalid}{\mathit{lvalid}}
\newcommand{\valid}{\mathit{valid}}
\newcommand{\tlc}{\ensuremath{\collecting{\mathit{tlc}}}}
\newcommand{\gprec}{\sqsubseteq} % greater precision
\newcommand{\sub}{<:}
\newcommand{\nsub}{<:_n}
\newcommand{\csub}{\lesssim}
\newcommand{\clsub}{\mathrel{\collecting{<:}}}
\newcommand{\ccsub}{\mathrel{\consistent{<:}}}
\newcommand{\join}{\sqcup}
\newcommand{\setdiff}{\mathrel{\backslash}}
\newcommand{\meet}{\sqcap}
\newcommand{\union}{\cup}
\newcommand{\setint}{\cap}
\newcommand{\finto}{\stackrel{\text{fin}}{\rightharpoonup}}% Finite partial map
\newcommand{\pfun}{\rightharpoonup}% partial function

% convention for consistent/gradual and collecting
\newcommand{\?}{\textsf{\upshape ?}} %unknown type
\newcommand{\consistent}[1]{\widetilde{#1}}
\newcommand{\collecting}[1]{\wideparen{#1}}
\newcommand{\cT}{{\consistent{T}}} %gradual type
\newcommand{\clT}{\collecting{T}}% collecting type
\newcommand{\ct}{\consistent{t}} %gradual term
\newcommand{\gvl}{\consistent{v}} %gradual value

% collecting / consistent relations
\newcommand{\carrow}{\;\collecting{\rightarrow}\;}
\newcommand{\clab}{\collecting{label}}
\newcommand{\ctype}{\collecting{type}}
\newcommand{\ccod}{\consistent{\cod}}
\newcommand{\cdom}{\consistent{\dom}}
\newcommand{\ccmb}{\consistent{\cmb}}
\newcommand{\clcmb}{\collecting{\cmb}}
\newcommand{\clcod}{\collecting{\cod}}
\newcommand{\cldom}{\collecting{\dom}}
\newcommand{\cP}{\consistent{P}}
\newcommand{\cF}{\consistent{F}}
\newcommand{\clpred}{\collecting{P}}
\newcommand{\cpred}{\cP}
\newcommand{\clfun}{\collecting{F}}
\newcommand{\cfun}{\consistent{F}}
\newcommand{\ceq}{\consistent{=}}
\newcommand{\cleq}{\collecting{=}}

%subtyping
\newcommand{\subjoin}{\mathbin{\begin{turn}{90}$\hspace{-0.2em}<:$\end{turn}}}
\newcommand{\submeet}{\mathbin{\begin{turn}{-90}$\hspace{-0.8em}<:$\end{turn}}}
\newcommand{\proj}{\mathit{proj}}
\newcommand{\cproj}{\consistent{\proj}}
\newcommand{\cssubmeet}{\mathbin{\consistent{\submeet}}}
\newcommand{\cssubjoin}{\mathbin{\consistent{\subjoin}}}

% security typing
\newcommand{\ltop}{\textsf{H}}
\newcommand{\lbot}{\textsf{L}}
\newcommand{\lx}{\ell} % label
\newcommand{\ul}{\?}% unknown label


\DeclareDocumentCommand{\bul}{ O{\lx_1} O{\lx_2} }{[{#1},{#2}]}
\newcommand{\bult}{\bul[\lx_3][\lx_4]}
\newcommand{\bull}[2]{\bul[{\lx_{#1}}][{\lx_{#2}}]}
\newcommand{\bulu}{\bul[\bot][\top]}

\newcommand{\clx}{{\tilde{\lx}}} % gradual label
%\newcommand{\clx}[1][]{{\tilde{\lx}\IfNoValueOrEmptyTF{#1}{}{_#1}}}} % gradual label
\newcommand{\cll}{\collecting{\lx}} % collecting label
\newcommand{\cS}{{\consistent{S}}} % gradual sec type
\newcommand{\clS}{\collecting{S}}% collecting sec type
\newcommand{\subl}{\preccurlyeq}% sub on labels

\newcommand{\clL}{\collecting{\sL}}% collecting ev label
\newcommand{\clE}{\collecting{\sE}}% collecting ev type
\newcommand{\sE}{{{E}}} % evidence type
\newcommand{\sL}{{{\lx}}} % list of labels
\newcommand{\cE}{{{\sE}}} % evidence type
\newcommand{\cL}{{\consistent{L}}} % list of labels
\newcommand{\csubl}{\;\consistent{\subl}\;}
\newcommand{\ljoincore}{\begin{turn}{90}$\hspace{-0.2em}\prec$\end{turn}}
\newcommand{\lmeetcore}{\begin{turn}{270}$\hspace{-0.6em}\prec$\end{turn}}
\newcommand{\ljoin}{\mathbin{\ljoincore}}
\newcommand{\lmeet}{\mathbin{\lmeetcore}}
\newcommand{\cjoin}{\mathbin{\consistent{\ljoincore}}}
\newcommand{\cjoinc}{\triangle}
\newcommand{\cmeet}{\mathbin{\consistent{\lmeetcore}}}
\newcommand{\cljoin}{\cjoin}
\newcommand{\clmeet}{\cmeet}
% gamma/alpha for labels/security types/pretypes
\newcommand{\gammal}{\gamma_\lx}
\newcommand{\alphal}{\alpha_\lx}
\newcommand{\gammas}{\gamma_S}
\newcommand{\gammat}{\gamma_T}
\newcommand{\alphas}{\alpha_S}
\newcommand{\alphat}{\alpha_T}
\newcommand{\slabel}{\mathit{label}}

\newcommand{\gammae}{{\gamma^{\tiny\ljoin}_E}}
\newcommand{\alphae}{{\alpha^{\tiny\ljoin}_E}}
\newcommand{\gammaem}{{\gamma^{\tiny\lmeet}_E}}
\newcommand{\alphaem}{{\alpha^{\tiny\lmeet}_E}}
\newcommand{\gammaL}{{\gamma^{\tiny\ljoin}_L}}
\newcommand{\gammaLM}{{\gamma^{\tiny\lmeet}_L}}
\newcommand{\gammaS}{{\gamma^{\tiny\star}_L}}
\newcommand{\alphaL}{{\alpha^{\tiny\ljoin}_L}}
\newcommand{\alphaLM}{{\alpha^{\tiny\lmeet}_L}}
% interior and evidence
\newcommand{\interior}[1]{\mathcal{I}_{#1}}
\DeclareDocumentCommand{\interiorf}{ m O{\joinfun}}{\mathcal{I}^{#2}_{#1}}

\newcommand{\Ev}[1]{\oblset{Ev}^{#1}}
\newcommand{\Isub}{\interior{<:}}
\newcommand{\Isubf}{\interiorf{<:}}
\newcommand{\Isubfm}{\interiorf{<:}[\lmeet]}
\newcommand{\trans}[1]{\circ^{#1}}
\renewcommand{\merge}[1]{\triangle^{#1}}

% intrinsic stuff
% \newcommand{\rec}[1]{\set{#1}}
\newcommand{\rec}[1]{[#1]}
\newcommand{\VarT}[1]{\oblset{Var}_{#1}}
\newcommand{\LocT}[1]{\oblset{Loc}_{#1}}
\newcommand{\TermT}[1]{\oblset{Term}_{#1}}
\newcommand{\injto}{\stackrel{\!\textrm{i\;f}}{\rightharpoonup}}
\newcommand{\EvTerm}{\oblset{EvTerm}}
\newcommand{\EvValue}{\oblset{EvValue}}
\newcommand{\EvLabel}{\oblset{EvLabel}}
\newcommand{\SimpleValue}{\oblset{SimpleValue}}
\newcommand{\EvFrame}{\oblset{EvFrame}}
\newcommand{\TmFrame}{\oblset{TmFrame}}
\newcommand{\pr}[1]{\braket{{#1}}}
\newcommand{\cast}[2]{\evcast{\evpr{#1}}{#2}}
\newcommand{\error}{\textup{\textbf{error}}}
\newcommand{\evt}{\mathit{et}}
\newcommand{\evv}{\mathit{ev}}
\newcommand{\red}{\longmapsto}
\newcommand{\nred}{~-->~}
%\newcommand{\lred}[1][\lxr]{\xmapsto{#1}}
%\newcommand{\lnred}[1][\lxr]{\xrightarrow{#1}}
\newcommand{\lred}[1][\lxr]{\overset{#1}{\red}}
\newcommand{\lnred}[1][\lxr]{\overset{#1}{\nred}}


% \newcommand{\cred}[1][]{~{\xmapsto{#1}}~}
% \newcommand{\cnred}[1][]{\overset{#1}{\nred}}
%\newcommand{\cred}[1][\el]{~{\overset{#1}{\red}}~}
%\newcommand{\cnred}[1][\el]{\overset{#1}{\nred}}

%\DeclareDocumentCommand{\cnred}{O{\el} O{\clxc}}{\overset{#1}{\nred}_{\IfNoValueOrEmptyTF{#2}{}{#2}}}
\DeclareDocumentCommand{\cred}{O{\el} O{\clxc}}{~\overset{#1}{\red}_{#2}}
\DeclareDocumentCommand{\cnred}{O{\el} O{\clxc}}{~\overset{#1}{-->}_{#2}~}

\newcommand{\credi}[0]{\cred[{\ev_{ri}\clxr_i}]}


\newcommand{\iapp}[1]{\mathbin{@^{#1}}}
\newcommand{\iappev}[3][{\evl}]{\mathbin{@^{{\color{staticcolor} #2}}_{{#1}}}}
\newcommand{\invcod}{\mathit{icod}}
\newcommand{\invdom}{\mathit{idom}}
\newcommand{\invproj}{\mathit{iproj}}

\newcommand{\sarray}[1]{
{\begin{array}[c]{@{}l@{}} #1
  \end{array}}}


% text
\newcommand{\ie}{\emph{i.e.}\xspace}
\newcommand{\etal}{\emph{et al.}\xspace}
\newcommand{\eg}{\emph{e.g.}\xspace}

% Annoying saveboxes because \wide* don't nest well
\newsavebox{\cTsave}
\savebox{\cTsave}{$\cT$}
\newsavebox{\clTsave}
\savebox{\clTsave}{$\clT$}

% Dynamic language (wrt simply-typed gradual language) terms
\newcommand{\DTerm}{\oblset{DTerm}}
\newcommand{\dyn}[1]{\check{#1}}
\newcommand{\dt}{\dyn{t}}

% Evidence
\newcommand{\ev}{\varepsilon}

\newcommand{\evl}[1][]{{{\ev_{\lx #1}}}}
% \newcommand{\evcast}[2]{#1#2}
% \newcommand{\evlangle}{\color{red}\langle}
% \newcommand{\evrangle}{\color{red}\rangle}
% \newcommand{\evpr}[1]{{\color{red}\langle}#1{\color{red}\rangle}}
\newcommand{\evcast}[2]{#1#2}
\newcommand{\evlangle}{\langle}
\newcommand{\evrangle}{\rangle}
\newcommand{\evpr}[1]{\braket{#1}}

\newcommand{\STFLsub}{\text{STFL}_{<:}}
\newcommand{\GTFLcsub}{\text{GTFL}_{\csub}}
%%%%%%%%%%%%%%%%%%%%%%%%%%%%%%%%%%%%%%%%%%%%%%%%%%%%%%%%%%%%%%%%%%%%%%

%REFS%
\newcommand{\braketeq}[2]{\braket{#1}}

\newcommand{\ConfigT}[1]{\oblset{Config}_{#1}}
\newcommand{\ConfigEl}[2]{#1|#2}

%LSECREF%
\newcommand{\new}[1]{\textcolor{memory}{#1}}
\newcommand{\Evidence}{\oblset{Evidence}}
% \newcommand{\lobs}{\zeta}
\newcommand{\lobs}{\lx_o}
\newcommand{\rel}[1][\lobs]{\approx_{#1}}
\newcommand{\rcomp}[1]{\mathcal{C}(#1)} 
\newcommand{\bval}[1]{bval(#1)}
\newcommand{\subst}{\sigma}
\newcommand{\itm}[1]{{t^{\cS_{#1}}}}
\renewcommand{\cast}[2]{#1#2}

\newcommand{\D}{\mathcal{D}}
\newcommand{\E}{\mathcal{E}}
 
\newcommand{\lateff}[1]{{\overset{#1}{-->}}}

\newcommand{\Reff}[0]{{\mathsf{Ref}}}
\newcommand{\Ref}[1]{{\mathsf{Ref}}~#1}
\newcommand{\Refl}[2]{{\mathsf{Ref}}_{#1}~#2}
\newcommand{\Unit}{\mathsf{Unit}}
\newcommand{\unit}{\mathsf{unit}}

%\DeclareDocumentCommand{\reff}{ m m O{\lxc} }{\new{\mathsf{ref}}^{#2,#3}~#1}
\newcommand{\reff}[1]{\mathsf{ref}~#1}
\DeclareDocumentCommand{\reffss}{ m m O{\lxc} }{{\mathsf{ref}}^{#2}~#1}
\DeclareDocumentCommand{\reffs}{ m m O{\clxc} }{{\mathsf{ref}}^{#2}~#1}
\DeclareDocumentCommand{\reffsev}{ m m O{\evl} O{\clxc} }{{{\mathsf{ref}}^{{\color{staticcolor} #2}}_{#3}}~#1}

\newcommand{\lett}[0]{\mathsf{let}}
\newcommand{\deref}[1]{{\mathsf{!}}#1}
\newcommand{\derefev}[2]{{\mathsf{!}}^{{\color{staticcolor} #1}} #2}
\newcommand{\assign}[2]{#1 {:=} #2}
\newcommand{\assignev}[3][{\evl}]{#2 \overset{#1}{:=} #3}
\newcommand{\tref}{\mathit{tref}}
\newcommand{\freeLocs}{\mathit{freeLocs}}
\newcommand{\ctref}{\consistent{\tref}}
\newcommand{\highlight}[1]{\textcolor{memory}{#1}}
\newcommand{\highlightr}[1]{\textcolor{red}{#1}}
\newcommand{\ConfT}[1]{\oblset{Conf}_{#1}}
\newcommand{\ConfEl}[2]{#1|#2}
\newcommand{\Store}[0]{\oblset{Store}}
\newcommand{\Pc}[0]{\oblset{Label}}
\newcommand{\cPc}[0]{\oblset{GLabel}}
\newcommand{\braketeqs}[1]{\braketeq{#1}{#1}}
\newcommand{\mudom}[0]{\mathit{dom}}
\newcommand{\invtref}[0]{\mathit{itref}} 
\newcommand{\true}{\mathsf{true}}
\newcommand{\false}{\mathsf{false}}
\newcommand{\ascOnHeap}[1]{#1}
\newcommand{\inn}{~\mathsf{in}}
\newcommand{\opt}[1]{}
\newcommand{\Loc}{\oblset{Loc}}
\newcommand{\loc}{{l}}
\newcommand{\lxlabel}{\mathsf{label}}
\newcommand{\clxlabel}{\consistent{label}}
%\newcommand{\lgsecref}{$\lambda^{{\text{REF}}}_{\consistent{\text{SEC}}}$\xspace}
\newcommand{\lgsecref}{GSL$_{\textsf{Ref}}$\xspace}
\newcommand{\siml}{~\approx~}
\newcommand{\invref}{\mathit{iref}}
\newcommand{\invlabel}{\mathit{ilbl}}
\newcommand{\invlat}{\mathit{ilat}}

\newcommand{\lambdaAn}[1]{\lambda^{{#1}}}
%\newcommand{\lambdaAn}[1]{\lambda^}
\newcommand{\lxc}[1][]{{\color{clxccolor} \lx_{c#1}}}
\newcommand{\lxcp}[1][]{{\color{clxccolor} \lx'_{c#1}}}
\newcommand{\lxcpp}[1][]{{\color{clxccolor} \lx''_{c#1}}}
\newcommand{\lxcppp}[1][]{{\color{clxccolor} \lx'''_{c#1}}}

\newcommand{\lxr}[1][]{{\color{dynamic} \lx_{r#1}}}
\newcommand{\clxc}[1][]{{\color{clxccolor} \tilde{\lx}_{c#1}}}
\newcommand{\clxcp}[1][]{{\color{clxccolor} \tilde{\lx}'_{c#1}}}
\newcommand{\clxcpp}[1][]{{\color{clxccolor} \tilde{\lx}''_{c#1}}}
\newcommand{\clxcppp}[1][]{{\color{clxccolor} \tilde{\lx}'''_{c#1}}}




\newcommand{\clxr}[1][]{{\color{dynamic} \tilde{\lx}_{r#1}}}
\newcommand{\clxrp}[1][]{{\color{dynamic} \tilde{\lx}'_{r#1}}}

\newcommand{\cllxc}{\collecting{\lxc}}
%\newcommand{\lsecref}{$\lambda_\text{SEC}^{\text{REF}}$\xspace}
\newcommand{\lsecref}{SSL$_{\textsf{Ref}}$\xspace}
\newcommand{\conf}[3]{#1 ~\new{| #2}}
\DeclareDocumentCommand{\iconf}{ m m O{\clxc} }{#2 \triangleright^{\hspace*{-0.5em}{}^{#3}}\hspace*{-0.4em} #1}
\DeclareDocumentCommand{\iconfnaive}{ m m O{\clxc} }{#2 \triangleright #1}



% \newcommand{\iconflg}[2]{\exists \clxc, \clxc |- #1}
\newcommand{\iconflg}[2]{\iconf{#1}{\evr\clxr[i]}[\clxc[i]]}

\DeclareDocumentCommand{\iconfr}{ m m m O{\clxc} }{\iconf{#2}{#1}[{#4}] ~{| #3}}
\DeclareDocumentCommand{\iconfre}{ m m m O{\clxc} }{#2~{| #3}}

\newcommand{\iconfrlg}[3]{\iconflg{#2}{#1} ~{| #3}}

\newcommand{\projL}[0]{\mathcal{L}}
\newcommand{\projLl}[1]{\mathcal{L}_{#1}}
\newcommand{\pH}[0]{\mathfrak{H}}
\newcommand{\wng}[1]{{\color{red}#1}}
\newcommand{\rval}{\mathit{rval}}
\newcommand{\prot}[2]{\mathsf{prot}_{#1}#2}

\DeclareDocumentCommand{\cprot}{ O{\evl\clx'} m m m O{\clxcp} O{\evrp}}{\mathsf{prot}^{{\color{staticcolor} #5,#2,#3}}_{#6,#1} #4}
\DeclareDocumentCommand{\cprots}{ O{\evl\clx'} m O{\evrp}}{\mathsf{prot}^{{\color{staticcolor}}}_{#3,#1} #2}
%\renewcommand{\mynote}[2]{}

\newcommand{\muwt}[2]{#1 |- #2}

%\newcommand{\logr}[3]{\braket{{\color{dynamic} #1}, #2, #3}}
\DeclareDocumentCommand{\logr}{ m m m O{\evr} O{\clxc}}{\braket{\iconf{#2}{\cast{#4}{{\color{dynamic} #1}}}[#5], #3}}
\DeclareDocumentCommand{\logrs}{ m m m O{\evr} O{\clxc}}{\braket{\cast{#4}{{\color{dynamic} #1}}, #5, #2, #3}}
\DeclareDocumentCommand{\logri}{ m m m m O{\evr} O{\clxc} O{}}{\braket{\iconf{#2_{#4}}{\cast{#5[#4]}{#1[#4]}}[#6[#4]], #3[#4]}}
\DeclareDocumentCommand{\logris}{ m m m m O{\evr} O{\clxc} O{}}{\braket{\cast{#5[#4]}{#1[#4]},#6[#4],#2_{#4}, #3[#4]}}
%\newcommand{\logr}[3]{\braket{\iconf{#1}{#2}, #3}}
\newcommand{\mulogr}[2]{\braket{{\color{dynamic} #1}, #2}}

\newcommand{\LTermT}[1]{\oblset{LTerm}_{#1}}
\newcommand{\lgrpc}[1][]{\clxr[#1]}

\newcommand{\uval}{\mathit{uval}}

\newcommand{\irred}{\mathit{irred}}

%ICFP2016-Refs%
\newcommand{\soplus}{\mathbin{\llbracket \oplus \rrbracket}}
\newcommand{\ncsubl}{\mathbin{\not \hspace{-2mm}\csubl}}
\newcommand{\nsubl}{\mathbin{\not\subl}}
\newcommand{\RawValue}{\oblset{RawValue}}
\mathlig{++}{\mathbin{++}}

\newcommand{\lref}{$\lambda^{\text{REF}}$\xspace}

\newcommand{\glref}{$\consistent{\lambda^{\text{REF}}}$\xspace}

\newmdenv[topline=false,bottomline=false,leftline=false]{borderright}

\newcommand{\permreff}[1]{\mathsf{ref^{*}}~{#1}}
\newcommand{\permRef}[1]{\mathsf{Ref^{*}}~{#1}}

\newcommand{\observable}[0]{\mathsf{obs}_{\lobs}}

\newcommand{\el}[1][]{{\color{dynamic} \mathit{e\lx}_{#1}}}
\newcommand{\StoreT}{{\Sigma}}


\DeclareDocumentCommand{\store}{ O{} }{ {\IfNoValueTF{#1}{\new{\mu}}{\new{\mu_{#1}}}}}
\DeclareDocumentCommand{\storep}{ O{} }{ {\IfNoValueTF{#1}{\new{\mu'}}{\new{\mu'_{#1}}}}}
\DeclareDocumentCommand{\storepp}{ O{} }{ {\IfNoValueTF{#1}{\new{\mu''}}{\new{\mu''_{#1}}}}}
\DeclareDocumentCommand{\storeppp}{ O{} }{ {\IfNoValueTF{#1}{\new{\mu'''}}{\new{\mu'''_{#1}}}}}

\newcommand{\finterval}[0]{\mathsf{Interval}}

\newcommand{\confs}[3]{#1 | #2}



\newcommand{\lnreds}[1][\lxr]{{\nred}}
\newcommand{\lreds}[1][\lxr]{~{\red}~}
\newcommand{\evr}[1][]{{\color{dynamic} \ev_{r#1}}}
\newcommand{\evrp}[1][]{{\color{dynamic} \ev'_{r#1}}}
\newcommand{\evrpp}[1][]{{\color{dynamic} \ev''_{r#1}}}

\newcommand{\joinfun}{\ljoin}
\newcommand{\joinfunE}{\ljoin_E}
\newcommand{\meetfun}{\lmeet}
\newcommand{\meetfunE}{\lmeet_E}
\newcommand{\cjoinfun}{\widetilde{\ljoin}}

\DeclareDocumentCommand{\setofjoins}{ m O{\joinfun} }{#2\overline{#1}}
\DeclareDocumentCommand{\setofjoinsE}{ m O{\joinfunE} }{#2\overline{#1}}
\DeclareDocumentCommand{\setofmeets}{ m O{\meetfun} }{#2\overline{#1}}
\DeclareDocumentCommand{\setofmeetsE}{ m O{\meetfunE} }{#2\overline{#1}}

\DeclareDocumentCommand{\setofcjoins}{m O{\cjoinfun}}{\setofjoins{#1}[#2]}

\DeclareDocumentCommand{\funkyset}{ O{\lx} O{\star} }{\setj{#1 \IfNoValueOrEmptyTF{#2}{}{, #2}}}
\DeclareDocumentCommand{\funkysetm}{ O{\lx} O{\star} }{\setm{#1 \IfNoValueOrEmptyTF{#2}{}{, #2}}}
%\DeclareDocumentCommand{\funkyset}{ O{\lx} O{*} }{\def\@tempa{#2} \set{#1 \ifx\@tempa\@empty\relax empty\else {, #2}\fi }}
\DeclareDocumentCommand{\funkysetu}{ }{ \funkyset[\ul][] }
\DeclareDocumentCommand{\funkysetum}{ }{ \funkysetm[\ul][] }


\newcommand{\eagerjoin}{\aries}
\newcommand{\eagermeet}{\mathrel{\reflectbox{\rotatebox[origin=c]{180}{$\eagerjoin$}}}}

\newcommand{\PC}{PC\xspace}
\newcommand{\meetjoin}{\meet^{\tiny\ljoin}}
\newcommand{\meetmeet}{\meet^{\tiny\lmeet}}

\DeclareDocumentCommand{\mergejoin}{ O{\subl} }{{\merge{#1}_{\tiny\ljoin}}}
\DeclareDocumentCommand{\mergemeet}{ O{\subl} }{\merge{#1}_{\tiny\lmeet}}

\DeclareDocumentCommand{\transjoin}{ O{\subl} }{{\trans{#1}_{\tiny\ljoin}}}
\DeclareDocumentCommand{\transmeet}{ O{\subl} }{{\trans{#1}_{\tiny\lmeet}}}
\newcommand{\evolution}{\rightarrowtriangle}
\newcommand{\labelset}{labelset\xspace}
\newcommand{\labelsets}{labelsets\xspace}


\newcommand{\setj}[1]{\set{#1}^{\tiny\ljoin}}
\newcommand{\setm}[1]{\set{#1}^{\tiny\lmeet}}
\newcommand{\sets}[1]{\set{#1}^{\tiny\star}}

\newcommand{\setofj}[1]{\setj{\overline{#1}}}
\newcommand{\setofm}[1]{\setm{\overline{#1}}}
\newcommand{\setofs}[1]{\sets{\overline{#1}}}


\newcommand{\thie}{FT\xspace}
\newcommand{\zdan}{ZD\xspace}

\newcommand{\clt}{t}

\newcommand{\clsLm}{{\collecting{\lx}^{\tiny\lmeet}}}
\newcommand{\clsLj}{{\collecting{\lx}^{\tiny\ljoin}}}

\newcommand{\latlx}[1][]{\lxc[#1]}
\newcommand{\latlxp}[1][]{\lxcp[#1]}
\newcommand{\latlxpp}[1][]{\lxcpp[#1]}
\newcommand{\latlxppp}[1][]{\lxcppp[#1]}

\newcommand{\clatlx}[1][]{\clxc[#1]}
\newcommand{\clatlxp}[1][]{\clxcp[#1]}
\newcommand{\clatlxpp}[1][]{\clxcpp[#1]}
\newcommand{\clatlxppp}[1][]{\clxcppp[#1]}
\newcommand{\unitifyd}{\mathit{uf}}
\newcommand{\refifyd}{\mathit{rf}}

\newcommand{\unitify}[1]{\unitifyd(#1)}
\newcommand{\refify}[1]{\refifyd(#1)}
\newcommand{\typetoevtype}{\Uparrow^{\cE}_{\cS}}
\newcommand{\emptyenv}{.}



\newcommand{\subljoin}{\mathbin{\begin{turn}{90}$\hspace{-0.2em}\subl$\end{turn}}}
\newcommand{\csubljoin}{\mathbin{\begin{turn}{90}$\hspace{-0.2em}\csubl$\end{turn}}}

\newcommand\Prefix[3]{\vphantom{#3}#1#2#3}

% \newcommand{\pra}[3]{{}^{{#2}}{\braket{{#1}}{}^{\hspace{-0.2em}#3}}}
\newcommand{\pra}[3]{\braket{#1}^{#2}_{#3}}
\newcommand{\cPa}[2]{\cP^{\tiny#1,#2}}

\newcommand{\joina}[2]{{\Prefix^{{#1}}{\subljoin^{{#2}}}}}
\newcommand{\cjoina}[2]{{\widetilde{\Prefix^{{#1}}{\subljoin^{{#2}}}}}}

\newcommand{\subjoina}[2]{{\Prefix^{{#1}}{\subjoin{}^{{#2}}}}}
\newcommand{\csubjoina}[2]{{\widetilde{\Prefix^{{#1}}{\subjoin{}^{{#2}}}}}}

% \newcommand{\subargs}[3]{{}^{{#2}}\overline{{#1}}{}^{{#3}}}
\newcommand{\subargs}[3]{\overline{#1}_{[{#2}/{#3}]}}

\newcommand{\relabel}{\mathit{relabel}}


\newcommand{\alphalj}[2]{\alpha'_{\lx}}
\newcommand{\alphalaj}{\alpha_{\evl}}
\newcommand{\alphalam}[2]{\alpha^{#1}_{#2}}
\newcommand{\alphalje}[2]{\alpha'^{#1,#2}_{\lx}}
\newcommand{\alphasj}[2]{\alpha'^{#1,#2}_{S}}
\newcommand{\alphasje}[2]{\alpha'^{#1,#2}_{S'}}
\newcommand{\alphasaj}{\alpha_{\ev}}
\newcommand{\alphasam}{\alpha_{\ev}}


\newcommand{\gammalj}[2]{\gamma'_{\lx}}
\newcommand{\gammalaj}{\gamma_{\evl}}
\newcommand{\gammalje}[2]{\gamma'^{#1,#2}_{\lx'}}
\newcommand{\gammasj}[2]{\gamma'^{#1,#2}_{S}}
\newcommand{\gammasje}[2]{\gamma'^{#1,#2}_{S'}}
\newcommand{\gammasaj}{\gamma_{\ev}}
\newcommand{\gammasam}{\gamma_{\ev}}



\newcommand{\setofaj}[5]{\set{\overline{\prap{#1}{#2}{#3}{#4}{#5}}}}
\newcommand{\setofam}[5]{\set{\overline{\prap{#1}{#2}{#3}{#4}{#5}}}}
\newcommand{\setofas}[5]{\set{\overline{\prap{#1}{#2}{#3}{#4}{#5}}}}

\newcommand{\setofop}[2]{{#2(\overline{#1})}}

\newcommand{\op}{o}
\newcommand{\opl}[1][]{o_{\lx #1}}
\newcommand{\ops}[1][]{o_{S #1}}
\newcommand{\Operation}{\oblset{Operation}}

\newcommand{\invjoin}{\ljoin^{-1}}
\newcommand{\invmeet}{\lmeet^{-1}}
\newcommand{\invsubjoin}{\subjoin^{-1}}
\newcommand{\invsubmeet}{\submeet^{-1}}

\newcommand{\eager}{\aries}
\newcommand{\eagerp}{\aries}

\newcommand{\concat}{c}


\newcommand{\prap}[5]{\braket{#1}^{#4#2}_{#5#3}}
\newcommand{\prp}[2]{\braket{#1}^{#2}}
\DeclareDocumentCommand{\interiorfp}{ m O{\opl[1]} O{\opl[2]}}{\mathcal{I}^{#2,#3}_{#1}}

\newcommand{\opstol}{\downarrow}

\newcommand{\alphaevl}{\alpha_{\evl}}
\newcommand{\gammaevl}{\gamma_{\evl}}
\newcommand{\alphaev}{\alpha_{\ev}}
\newcommand{\gammaev}{\gamma_{\ev}}
\newcommand{\alphaiv}{\alpha_{\iv}}
\newcommand{\gammaiv}{\gamma_{\iv}}

\newcommand{\alphaE}{\alpha_{\cE}}
\newcommand{\gammaE}{\gamma_{\cE}}

\newcommand{\iv}{\imath}

\newcommand{\Interval}{\Label^2}


\newcommand{\bounds}{\mathit{bounds}}

\newcommand{\trfunb}{\opstol}
\newcommand{\trfunbi}{\opstol^{\mathit{-1}}}
\newcommand{\trfunbis}[1]{#1^{-1}}
\newcommand{\trfunbim}[1]{#1^{\meet}}







%-\renewcommand{\mynote}[2]{}

\begin{document}

\begin{figure}
\begin{small}
%\begin{table}
\begin{center}
\begin{tabular}{|l|l|}
\hline
\begin{tabular}{l c l r}
%$\mathsf {Syntax}:$&&&\\
&&&\\
$t$&$::=$&&$\mathsf {terms}$\\
&&$\truet$&$\mathsf {constant \ true}$\\
&&$\falset$&$\mathsf {constant \ false}$\\
&&$\nvt$&$\mathsf {numeric \ value}$\\
&&$x$&$\mathsf {variable}$\\
&&$\abs {x}{T}{t}$&\ $\mathsf {abstraction}$\\
&&$\app {t_1}{t_2}$&$\mathsf {application}$\\
&&$\oletD$&$\mathsf {overloading \ let}$\\
&&$\ascrip {t}$&$\mathsf {ascription}$\\
&&&\\
$c$&$::=$&&$\mathsf {configurations}$\\
%&&$\conf {x}$&\\
%&&$\conf {\abs {x}{T}{t}}$&\\
%&&$\conf {\app {t_1}{t_2}}$&\\
&&$t[s]$&\\
&&$\ascrip{c}$&\\
&&$\oletP {T}{c}{c}$&\\
&&$c \ c $&\\
$v$&$::=$&&$\mathsf {values}$\\
&&$\truet[s]$&$\mathsf {true \ value}$\\
&&$\falset[s]$&$\mathsf {false \ value}$\\
&&$\nvt[s]$&$\mathsf {numeric \ value}$\\
&&$\conf {\abs {x}{T}{t}}$&$\mathsf {closure}$\\
&&&\\
&&&\\
\end{tabular}
& \begin{tabular}{l c l r}
$T$&$::=$&&$\mathsf {types}$\\
&&$\intt$&$\mathsf {type \ of \ integers}$\\
&&$\boolt$&$\mathsf {type \ of \ booleans}$\\
&&$T \to T$&$\mathsf {type \ of \ functions}$\\
&&&\\
$\mtP{T}$&$::=$&&$\mathsf {multi-types}$\\
&&$\mtC{T}$&$\mathsf {multi-type}$\\
&&&\\
$\Gamma$&$::=$&&$\mathsf {typing \ contexts}$\\
&&$\varnothing$&$\mathsf {empty \ context}$\\
&&$\Gamma , x:T$&$\mathsf {term \ variable \ binding}$\\
&&&\\
$\emt$&$::=$&&$\mathsf {multi-typing \ contexts}$\\
&&$\varnothing$&$\mathsf {empty \ context}$\\
&&$\emt,x: \mtP{T}$&$\mathsf {term \ variable \ binding}$\\
&&&\\
$s$&$::=$&&$\mathsf {explicit \ substitutions}$\\
&&$ \bullet$&$\mathsf {empty \ substitution}$\\
&&$x,\{(\overline {v:T})\}:s$&$\mathsf {variable \ substitution}$\\
\end{tabular}\\
\hline
\end{tabular}
\caption{Syntax of the simply typed lambda-calculus with overloading.}
\label{figure:sencilla}
\end{center}
%\end{table}
\end{small}
\end{figure}

\begin{figure}
\begin{small}
\begin{center}
\setlength{\tabcolsep}{1pt}
%\setlength{\extrarowheight}{20pt}
\hspace*{-4cm}
\begin{tabular}{|l|l|}

\hline
\begin{tabular}{c r}
&\\
\multicolumn{2}{r}{\framebox{$\env t \inferir \mtCu{T} \Rightarrow t$}}\\
&\\
$\env  \truet\inferir \boolt$&$\rulename{STTrue}$\\
&\\
$\env  \falset\inferir \boolt$&$\rulename{STFalse}$\\
&\\
$\inference {x:T \in \Gamma}{\env x \inferir \mtCu{T}}$&$\rulename{STVar\Gamma}$\\
&\\
$\inference {x:\mtD \in \emt}{\env x \inferir \mtD}$&$\rulename{STVar\emt}$\\
&\\
$\inference{\env t \chequear T}{\env \ascrip{t} \inferir \mtCu{T}}$&$\rulename{STAsc}$\\
&\\
$\inference{\env t_1 \inferir \mtP{T_1} &T_1 \in  \mtP{T_1}& \enve t_2 \inferir \mtP{T_2}}{\env \olet \inferir \mtP{T_2}}$&$\rulename{STOLet}$\\
&\\
$\inference {x \not \in dom(\Gamma \cup \emt) & \envE t_2 \inferir\mtP{T_2}}{\env \absD \inferir \norm{T_1 \to T_2}}$&$\rulename{STAbs}$\\
&\\
$\inference {\env t_1 \inferir \mtP{T} &  \exists! \ T_{1}\to T_{2} \in \mtP{T}| \env t_2 \chequear T_{1}}{\env \appD \inferir \mtCu{T_{2}}\Rightarrow (\ascripP{t_1}{T_1}) \ (\ascripP{t_2}{T_2})}$&$\rulename{STApp}$\\
&\\
\end{tabular}&
\begin{tabular}{c r}
&\\
\multicolumn{2}{r}{\framebox{$\env t \chequear T   \Rightarrow t$}}\\
&\\
$\env \truet \chequear \boolt$&$\rulename{CTTrue}$\\
&\\
$\env \falset \chequear \boolt$&$\rulename{CTFalse}$\\
&\\
$\inference {x:T \in \Gamma}{\env x \chequear T}$&$\rulename{CTVar\Gamma}$\\
&\\
$\inference {x:\mtD \in \emt & T \in \mtD}{\env x \chequear T \Rightarrow \ascrip{x}}$&$\rulename{CTVar\emt}$\\
&\\
$\inference{\env t \chequear T}{\env \ascrip{t} \chequear T}$&$\rulename{CTAsc}$\\
&\\
$\inference{\env t_1 \inferir \mtP{T_1} &T_1 \in  \mtP{T_1}& \enve t_2 \chequear T_2}{\env \olet \chequear T_2}$&$\rulename{CTOLet}$\\
&\\
$\inference {x \not \in dom(\Gamma \cup \emt) & \envE t_2 \chequear T_2}{\env \absD \chequear T_1 \to T_2}$&$\rulename{CTAbs}$\\
&\\
$\inference {\env t_1 \inferir \mtP{T} &  \exists! \ T_{1}\to T_{2} \in \mtP{T}| \env t_2 \chequear T_{1}}{\env \appD \chequear T_{2}\Rightarrow (\ascripP{t_1}{T_1}) \ (\ascripP{t_2}{T_2})}$&$\rulename{CTApp}$\\
&\\
\end{tabular}\\
\hline
\end{tabular}
\hspace*{-4cm}
\caption{Term synthesis and checking.}
\label{tabla:sencillaA}
\end{center}

\end{small}
\end{figure}

\begin{figure}
\begin{small}
\begin{center}
\setlength{\tabcolsep}{1pt}
%\setlength{\extrarowheight}{20pt}
\hspace*{-4cm}
\begin{tabular}{|l|l|}

\hline
\begin{tabular}{c r}
&\\
\multicolumn{2}{r}{\framebox{$\tyC c \inferir \mtCu{T} \Rightarrow c$}}\\
&\\
$\tyC \truet[s] \inferir \boolt$&$\rulename{STCTrue}$\\
&\\
$\tyC \falset[s] \inferir \boolt$&$\rulename{STCFalse}$\\
&\\
$\inference {x:T \in \Gamma}{\tyC x[s] \inferir \mtCu{T}}$&$\rulename{STCVar\Gamma}$\\
&\\
$\inference {(x,v) \in s & \tyC v \inferir \mtP{T}}{\tyC x[s] \inferir \mtD}$&$\rulename{STCVar\emt}$\\
&\\
$\inference{\tyC \ascrip{t[s]} \inferir \mtP{T}}{\tyC \conf{\ascrip{t}} \inferir \mtP{T}}$&$\rulename{STCAsc}$\\
&\\
$\inference{\tyC c \chequear T}{\tyC \ascrip{c} \inferir \mtCu{T}}$&$\rulename{STCCAsc}$\\
&\\
$\inference{\tyC \oletP{T_1}{t_1[s]}{t_2[s] \inferir \mtP{T}}}{\tyC \conf{\olet} \inferir \mtP{T}}$&$\rulename{STCOLet}$\\
&\\
$\inference{\tyC c_1 \inferir \mtP{T_1} &T_1 \in  \mtP{T_1}&  \enve  t_2  \inferir \mtP{T_2} }{\tyC \oletP{T_1}{c_1}{t_2[s]} \inferir \mtP{T_2}}$&$\rulename{STCCOLet}$\\
&\\
$\inference {x \not \in dom(\Gamma \cup \emt(s)) & \envEC t_2 \inferir\mtP{T_2}}{\tyC \conf {\absD} \inferir \norm{T_1 \to \mtP{T_2}}}$&$\rulename{STCAbs}$\\
&\\
$\inference {\tyC t_1[s] \ t_2[s] \inferir \mtP{T}}{\tyC \conf{\appD}\inferir \mtP{T}}$&$\rulename{STCApp}$\\
&\\
$\inference {\tyC c_1 \inferir \mtP{T} &  \exists! \ T_{1}\to T_{2} \in \mtP{T}| \tyC c_2 \chequear T_{1}}{\tyC c_1 \ c_2 \inferir \mtCu{T_{2}}\Rightarrow (\ascripP{c_1}{T_1}) \ (\ascripP{c_2}{T_2})}$&$\rulename{STCCApp}$\\
&\\
\end{tabular}&
\begin{tabular}{c r}
&\\
\multicolumn{2}{r}{\framebox{$\tyC c \chequear T \Rightarrow c$}}\\
&\\
$\tyC \truet[s] \chequear \boolt$&$\rulename{CTCTrue}$\\
&\\
$\tyC \falset[s] \chequear \boolt$&$\rulename{CTCFalse}$\\
&\\
$\inference {x:T \in \Gamma}{\tyC x[s] \chequear T}$&$\rulename{CTCVar\Gamma}$\\
&\\
$\inference {(x,v) \in s & \tyC v \inferir \mtP{T} & T \in \mtP{T}}{\tyC x[s] \chequear T \Rightarrow  \conf{\ascrip{x}}}$&$\rulename{CTCVar\emt}$\\
&\\
$\inference{\tyC \ascrip{t[s]} \chequear T}{\tyC \conf{\ascrip{t}} \chequear T}$&$\rulename{CTCAsc}$\\
&\\
$\inference{\tyC c \chequear T}{\tyC \ascrip{c} \chequear \mtCu{T}}$&$\rulename{CTCCAsc}$\\
&\\
$\inference{\tyC \oletP{T_1}{t_1[s]}{t_2[s] \chequear T}}{\tyC \conf{\olet} \chequear \mtP{T}}$&$\rulename{CTCOLet}$\\
&\\
$\inference{\tyC c_1 \inferir \mtP{T_1} &T_1 \in  \mtP{T_1}&  \enve  t_2 \chequear T_2}{\tyC \oletP{T_1}{c_1}{t_2[s]} \chequear T_2}$&$\rulename{CTCCOLet}$\\
&\\
$\inference {x \not \in dom(\Gamma \cup \emt(s)) & \envEC t_2 \chequear T_2}{\tyC \conf {\absD} \chequear T_1 \to T_2}$&$\rulename{CTCAbs}$\\
&\\
$\inference {\tyC t_1[s] \ t_2[s] \chequear T}{\tyC \conf{\appD}\chequear T}$&$\rulename{CTCApp}$\\
&\\
$\inference {\tyC c_1 \inferir \mtP{T} &  \exists! \ T_{1}\to T_{2} \in \mtP{T}| \tyC c_2 \chequear T_{1}}{\tyC c_1 \ c_2 \chequear T_{2} \Rightarrow (\ascripP{c_1}{T_1}) \ (\ascripP{c_2}{T_2})}$&$\rulename{CTCCApp}$\\
&\\
\end{tabular}\\
\hline
\end{tabular}
\hspace*{-4cm}
\caption{Configuration synthesis and checking.}
\label{tabla:sencillaA}
\end{center}

\end{small}
\end{figure}

\begin{figure}
\begin{small}
\begin{center}
\begin{tabular}{|c r|}
\hline
&\\
&\framebox {$c \tto c$}\\
&\\
$\confxu{x} \tto v$&$\rulename{VarOk}  $\\
&\\
$\inference {x \neq y} {\confy{x} \tto x[s]}$&$\rulename{VarNext}  $\\
&\\
$\inference {(v_i, T_{1i}) \in \{(\overline{v:T_1})\}}{\ascrip{\confx{x}} \tto v_i}$&$\rulename{VarAscOk}  $\\
&\\
$\inference {x \neq y}{\ascrip{\confy{x}} \tto \ascrip{x[s]}}$&$\rulename{VarAscNext}  $\\
&\\
$ \conf{\ascrip{t}} \tto \ascrip{t[s]}$&$\rulename{AscSub}$\\
&\\
$ \inference {c \tto c'}{\ascrip{c} \tto \ascrip{c'} }$&$\rulename{Asc} $\\
&\\
$ \ascrip{v} \tto v $&$\rulename{AscV} $\\
&\\
$ \conf{\olet} \tto \oletP{T_1}{t_1[s]}{t_2[s]}$&$\rulename{LetSub} $\\
&\\
${\oletP{T_1}{v}{t_2[s]} \tto \confx{t_2}}$&$\rulename{Let} $\\
&\\
$ \inference {c_1 \tto c_1'}{\oletP{T_1}{c_1}{t_2[s]} \tto \oletP{T_1}{c_1'}{t_2[s]} }$&$\rulename{Let1} $\\
&\\
$\conf{\appD} \tto t_1 [s] \ t_2 [s]$&$\rulename{AppSub} $\\
&\\
$\conf{\absD} \ v \tto \conf{[x \mapsto v]{t_2}}$&$\rulename{App}  $\\
&\\
$\inference {c_1 \tto c_1'}{c_1 \ c_2 \tto c_1' \ c_2} $&$\rulename{App1}  $\\
&\\
$ \inference {c \tto c'}{v  \ c \tto v \ c'}$&$\rulename{App2}  $\\
&\\
\hline
\end{tabular}
\caption{Configuration reduction rules.}
\label{tabla:sencilla}
\end{center}
\end{small}
\end{figure}

\begin{lemma}[Inversion of term typing]
\label{lemma:itt}
\mbox{}
\begin{enumerate}
\item If $\env x : R$ , then $x : R \in \Gamma$
\item If $\env \absD : R$, then $R = T_1 \to R_2$ for some $R_2$, with $\envE t_2 : R_2$.
\item If $\env t_1 \ t_2 : R$, then there is some type $T_{11}$ such that $\env t_1 : T_{11} \to R$ and $\env t_2 : T_{11}$.
\end{enumerate}
\end{lemma}
\begin{proof}
Immediate from the definition of the typing relation.
\end{proof}

\begin{lemma} [Inversion of configuration typing]
\label{lemma:ict}
\mbox{}
\begin{enumerate}
\item If $\tyC x[s] : R$ , then $(x,v) \in s$, for some $v$, and $\tyC v:R$.
\item If $\tyC \conf{\absD}: R$, then $R = T_1 \to R_2$ for some $R_2$, with $\envEC t_2 : R_2$.
\item If $\tyC  \conf{t_1 \ t_2} : R$, then $\tyC  t_1 [s] \ t_2 [s]:R$.
\item If $\tyC   c_1 \ c_2 : R$, then there is some type $T_{11}$ such that $\tyC c_1 : T_{11} \to R$ and $\tyC c_2: T_{11}$.
\end{enumerate}
\end{lemma}

\begin{proof}Immediate from the definition of the typing relation.
\end{proof}

\begin{lemma} [Canonical Forms]
\label{lemma:cf}
\mbox{}
\begin{enumerate}
\item If $v$ is a value of type $T_1 \to T_2$, then $v = \conf{\absD}$.
\end{enumerate}
\end{lemma}

\begin{proof} Straightforward.
\end{proof}

\begin{definition}[$\Gamma(s)$]
\label{definition:tcs}
\mbox{}
The typing context built from a substitution $s$, writing $\Gamma(s)$, it is defined as follows:
\[ \Gamma(s) = \begin{cases} 
     \varnothing & s =  \bullet \\
      \Gamma(s'), x:T & s = (x,v):s' \ \land \ \tyC v : T 
   \end{cases}
\]
\end{definition}

\begin{theorem}[Progress]
\label{theorem:progress}
\mbox{}
Suppose $c$ is a well-typed configuration (that is, $ \tyC c : T$ for some $T$). Then either $c$ is a value or else there is some $c'$ such that $c \tto c'$.
\end{theorem}

\begin{proof} By induction on a derivation of $\tyC c : T$.
\begin{case}[TCVar]
Then $c = x[s]$, with $(x,v) \in s$, for some $v$,  and $\tyC v:T$. Since $x \in dom(s)$, if the substitution $s = \Subx$, then rule $\textsl {VarOk}$, applies, otherwise, rule $\textsl {VarNext}$ applies.
\end{case}

\begin{case}[TCAbs]
 Then $c = \conf{\absD}$. This case is immediate, since closures are values.
\end{case}

\begin{case}[TCApp] 
 Then $c = \conf{t_1 \ t_2}$, so rule $\textsl {AppSub}$ applies to $c$.
\end{case}

\begin{case}[TCCApp]
 Then $c = c_1 \ c_2$, with $\tyC c_1 : T_{11} \to T$, for some $T_{11}$ and $\tyC c_2 : T_{11}$, by the Lemma~\ref{lemma:ict}. Then, by the induction hypothesis, either $c_1$ is a value or else it can take a step of evaluation, and likewise $c_2$. If $c_1$ can take a step, then rule $\textsl {App1}$ applies to $c$. If $c_1$ is a value and $c_2$ can take a step, then rule $\textsl {App2}$ applies. Finally, if both $c_1$ and $c_2$ are values, then the Lemma~\ref{lemma:cf} tells us that $c_1$ has the form $\conf{\lambda x: T_{11}.t_{12}}$, and so rule $\textsl {App}$ applies to $c$.
\end{case}
\end{proof}

\begin{definition}[Well typed substitution]
\label{definition:wts}
\mbox{}
A substitution $s$ is said well typed with a typing context $\Gamma$, writing $\env s$, if $dom(s) = dom(\Gamma)$ and for every $(x,v) \in s$ and $\tyC v:T$, where $x:T \in \Gamma$.
\end{definition}

\begin{lemma}[Permutation]
\label{lemma:permutation}
\mbox{}
If $\env t : T$ and $\Delta$ is a permutation of $\Gamma$, then $\Delta \vdash t : T$.
\end{lemma}

\begin{proof}By induction on typing derivations.
\end{proof}

\begin{lemma}
\label{lemma:1}
\mbox{}
If $\env s$ then $\Gamma$ is a permutation of $\Gamma(s)$.
\end{lemma}

\begin{proof}By the definition of well typed substitution.
\end{proof}
\begin{lemma}
\label{lemma:2}
\mbox{}
If $\env s$ and $\tyC v : T$, then $\Gamma,x:T \vdash \SubxD$.
\end{lemma}

\begin{proof} By the definition of well typed substitution.\end{proof}

\begin{lemma}
\label{lemma:3}
\mbox{}
If $\env s$ then $\tyC t[s] : T$ if and only if $\env t:T$.
\end{lemma}

\begin{proof}By induction on typing derivations, using Lemma~\ref{lemma:permutation} and Lemma~\ref{lemma:1}.
\end{proof}


\begin{theorem}[Preservation]
\label{theorem:preservation}
\mbox{}
If $\tyC c : T$ and $c \tto c'$, then $\tyC c	' : T$.
\end{theorem}

\begin{proof} By induction on a derivation of $\tyC c : T$.


\begin{case}[TCVar]
Then $c = x[s]$, with $\tyC (x,v) \in s$, for some $v$, and $\tyC v:T$. We find that there are two rule by which $c \tto c'$ can be derived: $\textsl {VarOk}$ and $\textsl {VarNext}$. We consider each case separately.
\begin{itemize}
\item $\textsl{Subcase}$ (VarOk). Then $s = \Subx$ and $c' = v$. Since $(x,v) \in s$ and $\tyC v:T$, then $\tyC c' : T$.

\item $\textsl{Subcase}$ (VarNext). Then $s = \Suby$, $x \neq y$ and $c' = x[s']$. Since $(x,v) \in s'$ too, and $\tyC v:T$ then $\tyC x[s']:T $, that is $\tyC c' : T$. 
\end{itemize}
\end{case}

\begin{case} [TAbs]
Then $c = \conf{\absD}$. It cannot be the case that $c \tto c'$, because $c$ is a value, then the requirements of the theorem are vacuously satisfied. 
\end{case}

\begin{case}[TCApp]
Then $c = \conf{t_1 \ t_2}$ and $\tyC t_1[s] \ t_2[s]: T$. We find that there are only one rule by which $c \tto c'$ can be derived: $\textsl {AppSub}$. With this rule $c' = t_1[s] \ t_2[s]$, then we can conclude that $\tyC c' : T$. 
\end{case}

\begin{case} [TCCApp]
Then $c = c_1 \ c_2$, $\tyC \ c_1 : T_2 \to T$ and $\tyC c_2 : T_2$. We find that there are three rules by which $c \tto c'$ can be derived: $\textsl {App1}$, $\textsl {App2}$ and $\textsl {App}$. We consider each case separately.
\begin{itemize}

\item $\textsl{Subcase}$ (App1). Then $c_1 \tto c_1', \ c' = c_1' \ c_2$. By the induction hypothesis, $\tyC c_1' : T_2 \to T$, then we can apply rule $TCCApp$, to conclude that $\tyC c_1' \ c_2: T$ , that is $\tyC c' : T$.

\item $\textsl{Subcase}$ (App2). Then $c_2 \tto c_2', \ c' = c_1 \ c_2'$. By the induction hypothesis, $\tyC c_2' : T_2 $, then we can apply rule $\textsl TCCApp$, to conclude that $\tyC c_1 \ c_2' : T$, that is $\tyC c' : T$.

\item $\textsl{Subcase}$ (App): Then $c_1 = \conf{\lambda x:T_{1}.t_{12}}$, $c_2 = v$, $c' = \confx{t_{12}}$ and $\envEC t_{12}: T$ by the Lemma~\ref{lemma:ict}. Since we know that $\envEC \SubxD$ by the Lemma~\ref{lemma:2}, the resulting configuration $\tyC \confx{t_{12}}:T$, by the Lemma~\ref{lemma:3}, that is $\vdash c' : T$.
\end{itemize}
\end{case}
\end{proof}
\end{document}
