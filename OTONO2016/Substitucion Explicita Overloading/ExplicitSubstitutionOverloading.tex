\documentclass[preprint,authoryear,sort&compress,9pt,nocopyrightspace]{article}

\usepackage{mathtools}
\usepackage{verbatim} % comentarios
\usepackage{definitions}

\newcommand\rulename[1]{\mathrm{(#1)}}
\newcommand{\nothing}{\varnothing} % different from \emptyset
\newcommand{\tto}{\longrightarrow}
\newcommand{\lto}{\leftarrow}

\newcommand{\conf}[2][s]{(#2)[#1]}
\newcommand{\confxu}[1]{#1 [x,\{(v:T_1)\}:s]}
\newcommand{\confxD}[1]{#1 [\subxD]}
\newcommand{\confx}[1]{#1 [\SubxD}
\newcommand{\confy}[1]{#1 [\SubyD]}
\newcommand{\confext}[1]{#1 [x,(v:T_1) \conc s]}
%\newcommand{\confxE}[1]{#1 [x,(v:T_1):s']}
%\newcommand{\confyE}[1]{#1 [y,(v:T_1):s']}


\newcommand{\subxD}{x \mapsto \{(\overline{v:T})\},s}
\newcommand{\SubxD}{x \mapsto \{(\overline{v:T_1})\},s}
\newcommand{\SubyD}{y \mapsto \{(\overline{v:T_1})\},s}
\newcommand{\Subx}{x \mapsto \{(\overline{v:T})\},s'}
\newcommand{\Suby}{y \mapsto \{(\overline{v:T})\},s'}
%\newcommand{\subx}{[x,(v:T_1):s']}
%\newcommand{\suby}{[y,(v:T_1):s']}

\newcommand{\tyC}{{\Gamma \vdash_c \ }}
\newcommand{\env}{{\Gamma ; \emt   \vdash \ }}

\newcommand{\envE}{{\Gamma , x:T_1  ; \emt \vdash \ }}
\newcommand{\enve}{{\Gamma ; \emt  \oplus (x : T_1)  \vdash \ }}

\newcommand{\envC}{{\Gamma ; \emt(s) \vdash \ }}
\newcommand{\envEC}{{\Gamma , x:T_1 ; \emt(s) \vdash \ }}
\newcommand{\enveC}{{\Gamma ;  \emt(s) \oplus (x : T_1) \vdash \ }}


\newcommand{\ascrip}[1]{#1::T}
\newcommand{\ascripP}[2]{#1::#2}
\newcommand{\oletD}{\mathsf{olet} \ x : T = t \ \mathsf{in}  \ t}
\newcommand{\olet}{\mathsf{olet} \ x : T_1 = t_1 \ \mathsf{in}  \ t_2}
\newcommand{\oletP}[3]{\mathsf{olet} \ x : #1 = #2 \ \mathsf{in}  \ #3}
\newcommand{\app}[2]{#1 \ #2}
\newcommand{\appD}{t_1 \ t_2}
\newcommand{\abs}[3]{\lambda #1:#2. \ #3}
\newcommand{\absD}{\lambda x:T_1. \ t_2}

\newcommand{\truet}{\mathsf{true}}
\newcommand{\falset}{\mathsf{false}}
\newcommand{\boolt}{\mathsf{Bool}}
\newcommand{\intt}{\mathsf{Int}}

\newcommand{\mtD}{T^{*}}
\newcommand{\mtP}[1]{#1^{*}}
\newcommand{\mtC}[1]{\{\overline {#1}\}}
\newcommand{\mtCu}[1]{\{ #1 \}}

\newcommand{\emt}{\phi}
\newcommand{\conc}{:^{*}}


\newcommand\inferir{\stackrel{\mathclap{\normalfont\mbox{$\to$}}}{\in}}
\newcommand\chequear{\stackrel{\mathclap{\normalfont\mbox{$\lto$}}}{\in}}
\newcommand\myeq{\stackrel{\mathclap{\normalfont\mbox{def}}}{=}}
\providecommand{\norm}[1]{\lVert#1\rVert}


\begin{document}

\begin{figure}
\begin{small}
%\begin{table}
\begin{center}
\begin{tabular}{|l|l|}
\hline
\begin{tabular}{l c l r}
&&&\\
$t$&$::=$&&$\mathsf {terms}$\\
&&$\truet$&$\mathsf {constant \ true}$\\
&&$\falset$&$\mathsf {constant \ false}$\\
&&$n$&$\mathsf {numeric \ value}$\\
&&$x$&$\mathsf {variable}$\\
&&$\abs {x}{T}{t}$&\ $\mathsf {abstraction}$\\
&&$\app {t_1}{t_2}$&$\mathsf {application}$\\
&&$\oletD$&$\mathsf {overloading \ let}$\\
&&$\ascrip {t}$&$\mathsf {ascription}$\\
&&&\\
$v$&$::=$&&$\mathsf {values}$\\
&&$\truet$&$\mathsf {true \ value}$\\
&&$\falset$&$\mathsf {false \ value}$\\
&&$n$&$\mathsf {numeric \ value}$\\
&&$\conf {\abs {x}{T}{t}}$&$\mathsf {closure}$\\
&&&\\
$c$&$::=$&&$\mathsf {configurations}$\\
&&$v$&\\
&&$t[s]$&\\
&&$\ascrip{c}$&\\
&&$\oletP {T}{c}{c}$&\\
&&$c \ c $&\\
\end{tabular}
& \begin{tabular}{l c l r}
$T$&$::=$&&$\mathsf {types}$\\
&&$\intt$&$\mathsf {type \ of \ integers}$\\
&&$\boolt$&$\mathsf {type \ of \ booleans}$\\
&&$T \to T$&$\mathsf {type \ of \ functions}$\\
&&&\\
$\mtP{T}$&$::=$&&$\mathsf {multi-types}$\\
&&$\mtC{T}$&$\mathsf {multi-type}$\\
&&&\\
$\Gamma$&$::=$&&$\mathsf {typing \ contexts}$\\
&&$\varnothing$&$\mathsf {empty \ context}$\\
&&$\Gamma , x:T$&$\mathsf {term \ variable \ binding}$\\
&&&\\
$\emt$&$::=$&&$\mathsf {multi-typing \ contexts}$\\
&&$\varnothing$&$\mathsf {empty \ context}$\\
&&$\emt,x: \mtP{T}$&$\mathsf {term \ variable \ binding}$\\
&&&\\
$s$&$::=$&&$\mathsf {explicit \ substitutions}$\\
&&$ \bullet$&$\mathsf {empty \ substitution}$\\
&&$\subxD$&$\mathsf {variable \ substitution}$\\
&&&\\
&&&\\
\end{tabular}\\
\hline
\end{tabular}
\caption{Syntax of the simply typed lambda-calculus with overloading.}
\label{figure:sencilla}
\end{center}
%\end{table}
\end{small}
\end{figure}

\begin{figure}
\begin{small}
\begin{center}
\setlength{\tabcolsep}{1pt}
%\setlength{\extrarowheight}{20pt}
\hspace*{-2cm}
\begin{tabular}{|l|l|}

\hline
\begin{tabular}{c r}
&\\
\multicolumn{2}{r}{\framebox{$\env t \inferir \mtD \Rightarrow t$}}\\
&\\
$\env  \truet\inferir \mtCu {\boolt}$&$\rulename{STTrue}$\\
&\\
$\env  \falset\inferir \mtCu{\boolt}$&$\rulename{STFalse}$\\
&\\
$\env  n \inferir \mtCu {\intt}$&$\rulename{STNum}$\\
&\\
$\inference {x:T \in \Gamma}{\env x \inferir \mtCu{T}}$&$\rulename{STVar\Gamma}$\\
&\\
$\inference {x:\mtD \in \emt}{\env x \inferir \mtD}$&$\rulename{STVar\emt}$\\
&\\
$\inference{\env t \chequear T}{\env \ascrip{t} \inferir \mtCu{T}}$&$\rulename{STAsc}$\\
&\\
$\inference{x \not \in dom(\Gamma) & \env t_1 \inferir \mtP{T_1}  & \\ T_1 \in  \mtP{T_1}& \enve t_2 \inferir \mtP{T_2}}{\env \olet \inferir \mtP{T_2}}$&$\rulename{STOLet}$\\
&\\
$\inference {x \not \in dom(\Gamma \cup \emt) & \envE t_2 \inferir\mtP{T_2}}{\env \absD \inferir \norm{T_1 \to T_2}}$&$\rulename{STAbs}$\\
&\\
$\inference {\env t_1 \inferir \mtP{T} \\ &  \exists! \ T_{1}\to T_{2} \in \mtP{T}| \env t_2 \chequear T_{1}}{\env \appD \inferir \mtCu{T_{2}}\Rightarrow (\ascripP{t_1}{T_1 \to T_2}) \ (\ascripP{t_2}{T_1})}$&$\rulename{STApp}$\\
&\\
\end{tabular}&
\begin{tabular}{c r}
&\\
\multicolumn{2}{r}{\framebox{$\env t \chequear T   \Rightarrow t$}}\\
&\\
$\env \truet \chequear \boolt$&$\rulename{CTTrue}$\\
&\\
$\env \falset \chequear \boolt$&$\rulename{CTFalse}$\\
&\\
$\env  n \chequear \intt$&$\rulename{CTNum}$\\
&\\
$\inference {\env x \inferir \mtCu{T}}{\env x \chequear T}$&$\rulename{CTVar\Gamma}$\\
&\\
$\inference {\env x \inferir \mtD & T \in \mtD}{\env x \chequear T \Rightarrow \ascrip{x}}$&$\rulename{CTVar\emt}$\\
&\\
$\inference{\env t \chequear T}{\env \ascrip{t} \chequear T}$&$\rulename{CTAsc}$\\
&\\
$\inference{x \not \in dom(\Gamma) & \env t_1 \inferir \mtP{T_1}  & \\ T_1 \in  \mtP{T_1}& \enve t_2 \chequear T_2}{\env \olet \chequear T_2}$&$\rulename{CTOLet}$\\
&\\
$\inference {x \not \in dom(\Gamma \cup \emt) & \envE t_2 \chequear T_2}{\env \absD \chequear T_1 \to T_2}$&$\rulename{CTAbs}$\\
&\\
$\inference {\env t_1 \inferir \mtP{T} \\ &  \exists! \ T_{1}\to T_{2} \in \mtP{T}| \env t_2 \chequear T_{1}}{\env \appD \chequear T_{2}\Rightarrow (\ascripP{t_1}{T_1 \to T_2}) \ (\ascripP{t_2}{T_1})}$&$\rulename{CTApp}$\\
&\\
\end{tabular}\\
\hline
\end{tabular}
\hspace*{-2cm}
\caption{Type synthesis and checking.}
%\caption{Term synthesis and checking.}
\label{tabla:sencillaA}
\end{center}

\end{small}
\end{figure}

\begin{figure}
\begin{small}
\begin{center}
\setlength{\tabcolsep}{1pt}
%\setlength{\extrarowheight}{20pt}
\hspace*{-3cm}
\begin{tabular}{|l|l|}

\hline
\begin{tabular}{c r}
&\\
\multicolumn{2}{r}{\framebox{$\tyC c \inferir \mtD \Rightarrow c$}}\\
&\\
$\tyC \truet \inferir \mtCu{\boolt}$&$\rulename{STCTrue}$\\
&\\
$\tyC \falset \inferir \mtCu{\boolt}$&$\rulename{STCFalse}$\\
&\\
$\tyC \truet[s] \inferir \mtCu{\boolt}$&$\rulename{STCCTrue}$\\
&\\
$\tyC \falset[s] \inferir \mtCu{\boolt}$&$\rulename{STCCFalse}$\\
&\\
$\tyC n \inferir \mtCu{\intt}$&$\rulename{STCNum}$\\
&\\
$\tyC n[s] \inferir \mtCu{\intt}$&$\rulename{STCCNum}$\\
&\\
$\inference {x:T \in \Gamma}{\tyC x[s] \inferir \mtCu{T}}$&$\rulename{STCVar\Gamma}$\\
&\\
%$\inference {x,\{(\overline{v:T})\} \in s & \tyC v \inferir \mtC{T}}{\tyC x[s] \inferir \mtC{T}}$&$\rulename{STCVar\emt}$\\
$\inference {x,\{(\overline{v:T})\} \in s}{\tyC x[s] \inferir \mtC{T}}$&$\rulename{STCVar\emt}$\\
&\\
$\inference{\tyC \ascrip{t[s]} \inferir \mtP{T}}{\tyC \conf{\ascrip{t}} \inferir \mtP{T}}$&$\rulename{STCAsc}$\\
&\\
$\inference{\tyC c \chequear T}{\tyC \ascrip{c} \inferir \mtCu{T}}$&$\rulename{STCCAsc}$\\
&\\
$\inference{\tyC \oletP{T_1}{t_1[s]}{t_2[s] \inferir \mtP{T}}}{\tyC \conf{\olet} \inferir \mtP{T}}$&$\rulename{STCOLet}$\\
&\\
$\inference{x \not \in dom(\Gamma) & \tyC c_1 \inferir \mtP{T_1}  & \\ T_1 \in  \mtP{T_1}  &   \enveC  t_2  \inferir \mtP{T_2} }{\tyC \oletP{T_1}{c_1}{t_2[s]} \inferir \mtP{T_2}}$&$\rulename{STCCOLet}$\\
&\\
$\inference {x \not \in dom(\Gamma \cup \emt(s)) & \envEC t_2 \inferir\mtP{T_2}}{\tyC \conf {\absD} \inferir \norm{T_1 \to \mtP{T_2}}}$&$\rulename{STCAbs}$\\
&\\
$\inference {\tyC t_1[s] \ t_2[s] \inferir \mtP{T}}{\tyC \conf{\appD}\inferir \mtP{T}}$&$\rulename{STCApp}$\\
&\\
$\inference {\tyC c_1 \inferir \mtP{T}  \\ &  \exists! \ T_{1}\to T_{2} \in \mtP{T}| \tyC c_2 \chequear T_{1}}{\tyC c_1 \ c_2 \inferir \mtCu{T_{2}}\Rightarrow (\ascripP{c_1}{T_1 \to T_2}) \ (\ascripP{c_2}{T_1})}$&$\rulename{STCCApp}$\\
&\\
\end{tabular}&
\begin{tabular}{c r}
&\\
\multicolumn{2}{r}{\framebox{$\tyC c \chequear T \Rightarrow c$}}\\
&\\
$\tyC \truet \chequear \boolt$&$\rulename{CTCTrue}$\\
&\\
$\tyC \falset \chequear \boolt$&$\rulename{CTCFalse}$\\
&\\
$\tyC \truet[s] \chequear \boolt$&$\rulename{CTCTrue}$\\
&\\
$\tyC \falset[s] \chequear \boolt$&$\rulename{CTCFalse}$\\
&\\
$\tyC n \chequear \intt$&$\rulename{CTCNum}$\\
&\\
$\tyC n[s] \chequear \intt$&$\rulename{CTCCNum}$\\
&\\
$\inference {\tyC x[s] \inferir \mtCu{T}}{\tyC x[s] \chequear T}$&$\rulename{CTCVar\Gamma}$\\
&\\
%$\inference {x,\{(\overline{v:T})\} \in s & \tyC v \inferir \mtP{T} & T \in \mtP{T}}{\tyC x[s] \chequear T \Rightarrow  \conf{\ascrip{x}}}$&$\rulename{CTCVar\emt}$\\
$\inference {x,\{(\overline{v:T})\} \in s & T \in \mtC{T}}{\tyC x[s] \chequear T \Rightarrow  \conf{\ascrip{x}}}$&$\rulename{CTCVar\emt}$\\
&\\
$\inference{\tyC \ascrip{t[s]} \chequear T}{\tyC \conf{\ascrip{t}} \chequear T}$&$\rulename{CTCAsc}$\\
&\\
$\inference{\tyC c \chequear T}{\tyC \ascrip{c} \chequear \mtCu{T}}$&$\rulename{CTCCAsc}$\\
&\\
$\inference{\tyC \oletP{T_1}{t_1[s]}{t_2[s] \chequear T}}{\tyC \conf{\olet} \chequear \mtP{T}}$&$\rulename{CTCOLet}$\\
&\\
$\inference{x \not \in dom(\Gamma) & \tyC c_1 \inferir \mtP{T_1} & \\T_1 \in  \mtP{T_1}&  \enveC  t_2 \chequear T_2}{\tyC \oletP{T_1}{c_1}{t_2[s]} \chequear T_2}$&$\rulename{CTCCOLet}$\\
&\\
$\inference {x \not \in dom(\Gamma \cup \emt(s)) & \envEC t_2 \chequear T_2}{\tyC \conf {\absD} \chequear T_1 \to T_2}$&$\rulename{CTCAbs}$\\
&\\
$\inference {\tyC t_1[s] \ t_2[s] \chequear T}{\tyC \conf{\appD}\chequear T}$&$\rulename{CTCApp}$\\
&\\
$\inference {\tyC c_1 \inferir \mtP{T} \\ &  \exists! \ T_{1}\to T_{2} \in \mtP{T}| \tyC c_2 \chequear T_{1}}{\tyC c_1 \ c_2 \chequear T_{2} \Rightarrow (\ascripP{c_1}{T_1 \to T_2}) \ (\ascripP{c_2}{T_1})}$&$\rulename{CTCCApp}$\\
&\\
\end{tabular}\\
\hline
\end{tabular}
\hspace*{-3cm}
\caption{Configuration synthesis and checking.}
\label{tabla:sencillaA}
\end{center}

\end{small}
\end{figure}

\begin{figure}
\begin{small}
\begin{center}
\begin{tabular}{|c r|}
\hline
&\\
&\framebox {$c \tto c$}\\
&\\
$\truet[s] \tto \truet$&$\rulename{True}  $\\
&\\
$\falset[s] \tto \falset$&$\rulename{False}  $\\
&\\
$n[s] \tto n$&$\rulename{Num}  $\\
&\\
$\confxu{x} \tto v$&$\rulename{VarOk}  $\\
&\\
$\inference {x \neq y} {\confy{x} \tto x[s]}$&$\rulename{VarNext}  $\\
&\\
$\inference {(v_i, T_i) \in \{(\overline{v:T})\}}{\ascripP{\confxD{x}}{T_i} \tto v_i}$&$\rulename{VarAscOk}  $\\
&\\
$\inference {x \neq y}{\ascrip{\confy{x}} \tto \ascrip{x[s]}}$&$\rulename{VarAscNext}  $\\
&\\
$ \conf{\ascrip{t}} \tto \ascrip{t[s]}$&$\rulename{AscSub}$\\
&\\
$ \ascrip{v} \tto v $&$\rulename{Asc} $\\
&\\
$ \inference {c \tto c'}{\ascrip{c} \tto \ascrip{c'} }$&$\rulename{Asc1} $\\
&\\
$ \conf{\olet} \tto \oletP{T_1}{t_1[s]}{t_2[s]}$&$\rulename{LetSub} $\\
&\\
${\oletP{T_1}{v}{t_2[s]} \tto \confext{t_2}}$&$\rulename{Let} $\\
&\\
$ \inference {c_1 \tto c_1'}{\oletP{T_1}{c_1}{t_2[s]} \tto \oletP{T_1}{c_1'}{t_2[s]} }$&$\rulename{Let1} $\\
&\\
$\conf{\appD} \tto t_1 [s] \ t_2 [s]$&$\rulename{AppSub} $\\
&\\
$\conf{\absD} \ v \tto \conf{[x \mapsto v]{t_2}}$&$\rulename{App}  $\\
&\\
$\inference {c_1 \tto c_1'}{c_1 \ c_2 \tto c_1' \ c_2} $&$\rulename{App1}  $\\
&\\
$ \inference {c \tto c'}{v  \ c \tto v \ c'}$&$\rulename{App2}  $\\
&\\
\hline
\end{tabular}
\caption{Configuration reduction rules.}
\label{tabla:sencilla}
\end{center}
\end{small}
\end{figure}

\begin{lemma}[Inversion of term typing]
\label{lemma:itt}
\mbox{}
\begin{enumerate}
\item If $\env x : R$ , then $x : R \in \Gamma$
\item If $\env \absD : R$, then $R = T_1 \to R_2$ for some $R_2$, with $\envE t_2 : R_2$.
\item If $\env t_1 \ t_2 : R$, then there is some type $T_{11}$ such that $\env t_1 : T_{11} \to R$ and $\env t_2 : T_{11}$.
\end{enumerate}
\end{lemma}
\begin{proof}
Immediate from the definition of the typing relation.
\end{proof}

\begin{lemma} [Inversion of configuration typing]
\label{lemma:ict}
\mbox{}
\begin{enumerate}
\item If $\tyC x[s] : R$ , then $(x,v) \in s$, for some $v$, and $\tyC v:R$.
\item If $\tyC \conf{\absD}: R$, then $R = T_1 \to R_2$ for some $R_2$, with $\envEC t_2 : R_2$.
\item If $\tyC  \conf{t_1 \ t_2} : R$, then $\tyC  t_1 [s] \ t_2 [s]:R$.
\item If $\tyC   c_1 \ c_2 : R$, then there is some type $T_{11}$ such that $\tyC c_1 : T_{11} \to R$ and $\tyC c_2: T_{11}$.
\end{enumerate}
\end{lemma}

\begin{proof}Immediate from the definition of the typing relation.
\end{proof}

\begin{lemma} [Canonical Forms]
\label{lemma:cf}
\mbox{}
\begin{enumerate}
\item If $v$ is a value of type $T_1 \to T_2$, then $v = \conf{\absD}$.
\end{enumerate}
\end{lemma}

\begin{proof} Straightforward.
\end{proof}

\begin{definition}[$\Gamma(s)$]
\label{definition:tcs}
\mbox{}
The typing context built from a substitution $s$, writing $\Gamma(s)$, it is defined as follows:
\[ \Gamma(s) = \begin{cases} 
     \varnothing & s =  \bullet \\
      \Gamma(s'), x:T & s = (x,v):s' \ \land \ \tyC v : T 
   \end{cases}
\]
\end{definition}

\begin{definition}[$\oplus$]
\label{definition:tcs}
\mbox{}
Given a multi-type context $\emt$ and a pair $(x:T)$, the operator $\oplus$ is defined  as follows:
\[ \emt \oplus (x:T) = \begin{cases} 
      x:\mtCu{T}& \emt =  \varnothing \\
      \emt',x:(\mtD \cup \mtCu{T}) & \emt = \emt',x:\mtD\\
      \emt' \oplus (x:T),y:\mtD & \emt = \emt',y:\mtD\\
   \end{cases}
\]
\end{definition}

\begin{definition}[$\norm{\cdotp}$]
\label{definition:tcs}
\mbox{}
Given $T_1 \to \mtP{T_2}$, the operator $\norm{\cdotp}$ is defined  as follows:
\[ \norm{T_1 \to \mtP{T_2}} = \{T_1 \to T_2|T_2 \in \mtP{T_2}\}
\]
\end{definition}

\begin{theorem}[Progress]
\label{theorem:progress}
\mbox{}
Suppose $c$ is a well-typed configuration (that is, $ \tyC c : T$ for some $T$). Then either $c$ is a value or else there is some $c'$ such that $c \tto c'$.
\end{theorem}

\begin{proof} By induction on a derivation of $\tyC c : T$.
\begin{case}[TCVar]
Then $c = x[s]$, with $(x,v) \in s$, for some $v$,  and $\tyC v:T$. Since $x \in dom(s)$, if the substitution $s = \Subx$, then rule $\textsl {VarOk}$, applies, otherwise, rule $\textsl {VarNext}$ applies.
\end{case}

\begin{case}[TCAbs]
 Then $c = \conf{\absD}$. This case is immediate, since closures are values.
\end{case}

\begin{case}[TCApp] 
 Then $c = \conf{t_1 \ t_2}$, so rule $\textsl {AppSub}$ applies to $c$.
\end{case}

\begin{case}[TCCApp]
 Then $c = c_1 \ c_2$, with $\tyC c_1 : T_{11} \to T$, for some $T_{11}$ and $\tyC c_2 : T_{11}$, by the Lemma~\ref{lemma:ict}. Then, by the induction hypothesis, either $c_1$ is a value or else it can take a step of evaluation, and likewise $c_2$. If $c_1$ can take a step, then rule $\textsl {App1}$ applies to $c$. If $c_1$ is a value and $c_2$ can take a step, then rule $\textsl {App2}$ applies. Finally, if both $c_1$ and $c_2$ are values, then the Lemma~\ref{lemma:cf} tells us that $c_1$ has the form $\conf{\lambda x: T_{11}.t_{12}}$, and so rule $\textsl {App}$ applies to $c$.
\end{case}
\end{proof}

\begin{definition}[Well typed substitution]
\label{definition:wts}
\mbox{}
A substitution $s$ is said well typed with a typing context $\Gamma$, writing $\env s$, if $dom(s) = dom(\Gamma)$ and for every $(x,v) \in s$ and $\tyC v:T$, where $x:T \in \Gamma$.
\end{definition}

\begin{lemma}[Permutation]
\label{lemma:permutation}
\mbox{}
If $\env t : T$ and $\Delta$ is a permutation of $\Gamma$, then $\Delta \vdash t : T$.
\end{lemma}

\begin{proof}By induction on typing derivations.
\end{proof}

\begin{lemma}
\label{lemma:1}
\mbox{}
If $\env s$ then $\Gamma$ is a permutation of $\Gamma(s)$.
\end{lemma}

\begin{proof}By the definition of well typed substitution.
\end{proof}
\begin{lemma}
\label{lemma:2}
\mbox{}
If $\env s$ and $\tyC v : T$, then $\Gamma,x:T \vdash \SubxD$.
\end{lemma}

\begin{proof} By the definition of well typed substitution.\end{proof}

\begin{lemma}
\label{lemma:3}
\mbox{}
If $\env s$ then $\tyC t[s] : T$ if and only if $\env t:T$.
\end{lemma}

\begin{proof}By induction on typing derivations, using Lemma~\ref{lemma:permutation} and Lemma~\ref{lemma:1}.
\end{proof}


\begin{theorem}[Preservation]
\label{theorem:preservation}
\mbox{}
If $\tyC c : T$ and $c \tto c'$, then $\tyC c	' : T$.
\end{theorem}

\begin{proof} By induction on a derivation of $\tyC c : T$.


\begin{case}[TCVar]
Then $c = x[s]$, with $\tyC (x,v) \in s$, for some $v$, and $\tyC v:T$. We find that there are two rule by which $c \tto c'$ can be derived: $\textsl {VarOk}$ and $\textsl {VarNext}$. We consider each case separately.
\begin{itemize}
\item $\textsl{Subcase}$ (VarOk). Then $s = \Subx$ and $c' = v$. Since $(x,v) \in s$ and $\tyC v:T$, then $\tyC c' : T$.

\item $\textsl{Subcase}$ (VarNext). Then $s = \Suby$, $x \neq y$ and $c' = x[s']$. Since $(x,v) \in s'$ too, and $\tyC v:T$ then $\tyC x[s']:T $, that is $\tyC c' : T$. 
\end{itemize}
\end{case}

\begin{case} [TAbs]
Then $c = \conf{\absD}$. It cannot be the case that $c \tto c'$, because $c$ is a value, then the requirements of the theorem are vacuously satisfied. 
\end{case}

\begin{case}[TCApp]
Then $c = \conf{t_1 \ t_2}$ and $\tyC t_1[s] \ t_2[s]: T$. We find that there are only one rule by which $c \tto c'$ can be derived: $\textsl {AppSub}$. With this rule $c' = t_1[s] \ t_2[s]$, then we can conclude that $\tyC c' : T$. 
\end{case}

\begin{case} [TCCApp]
Then $c = c_1 \ c_2$, $\tyC \ c_1 : T_2 \to T$ and $\tyC c_2 : T_2$. We find that there are three rules by which $c \tto c'$ can be derived: $\textsl {App1}$, $\textsl {App2}$ and $\textsl {App}$. We consider each case separately.
\begin{itemize}

\item $\textsl{Subcase}$ (App1). Then $c_1 \tto c_1', \ c' = c_1' \ c_2$. By the induction hypothesis, $\tyC c_1' : T_2 \to T$, then we can apply rule $TCCApp$, to conclude that $\tyC c_1' \ c_2: T$ , that is $\tyC c' : T$.

\item $\textsl{Subcase}$ (App2). Then $c_2 \tto c_2', \ c' = c_1 \ c_2'$. By the induction hypothesis, $\tyC c_2' : T_2 $, then we can apply rule $\textsl TCCApp$, to conclude that $\tyC c_1 \ c_2' : T$, that is $\tyC c' : T$.

\item $\textsl{Subcase}$ (App): Then $c_1 = \conf{\lambda x:T_{1}.t_{12}}$, $c_2 = v$, $c' = \confx{t_{12}}$ and $\envEC t_{12}: T$ by the Lemma~\ref{lemma:ict}. Since we know that $\envEC \SubxD$ by the Lemma~\ref{lemma:2}, the resulting configuration $\tyC \confx{t_{12}}:T$, by the Lemma~\ref{lemma:3}, that is $\vdash c' : T$.
\end{itemize}
\end{case}
\end{proof}
\end{document}
