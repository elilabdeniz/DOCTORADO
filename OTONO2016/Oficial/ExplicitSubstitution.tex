\documentclass {article}


\usepackage{changepage}
\usepackage{lipsum}
\usepackage{caption}
\usepackage{amsmath}
\usepackage{proof}
\usepackage[T1]{fontenc}
\usepackage{ebproof}
\usepackage{amssymb} % To provide the \varnothing symbol
\usepackage[latin1]{inputenc} % acentos sin codigo
\usepackage{verbatim} % comentarios
\usepackage[spanish]{babel}
\usepackage{mathtools}
\usepackage{setspace}
\usepackage{soul}%sombreado
\usepackage[pdftex]{color}
\usepackage{semantic}
\usepackage{float}
\usepackage{definitions}
\usepackage{array}
\usepackage{longtable}
\usepackage{xcolor,colortbl}
\usepackage{multirow}
\usepackage[letterpaper]{geometry}

\newcommand{\nvector}[2][a]{#1_{1},#1_{2},#1_{3},\cdots #1_{#2}}
\newcommand{\vect}{(x_1,x_2,\dots,x_n)}
\newcommand\gij[3]{\Gamma \vdash #1 \Rightarrow #2 : #3 }
\newcommand\rulename[1]{\mathrm{(#1)}}
\newcommand{\nothing}{\varnothing} % different from \emptyset
\newcommand{\tto}{\longrightarrow}
\newcommand{\lambdax}{\lambda x}
\newcommand\myeq{\stackrel{\mathclap{\normalfont\mbox{def}}}{=}}

\newcommand{\app}[2]{#1 \ #2}
\newcommand{\appD}{t_1 \ t_2}
\newcommand{\abs}[3]{\lambda #1:#2. \ #3}
\newcommand{\absD}{\lambda x:T_1. \ t_2}
\newcommand{\conf}[2][s]{#2 \ [#1]}
\newcommand{\confx}[1]{#1 \ [(x,v):s]}
\newcommand{\confy}[1]{#1 \ [(y,v):s]}
\newcommand{\confxE}[1]{#1 \ [(x,v):s']}
\newcommand{\confyE}[1]{#1 \ [(y,v):s']}
\newcommand{\subxD}{[(x,v):s]}
\newcommand{\subx}{[(x,v):s']}
\newcommand{\suby}{[(y,v):s']}
\newcommand{\env}{{\Gamma \vdash \ }}
\newcommand{\envE}{{\Gamma , x:T_1 \vdash \ }}
\newcommand{\envC}{{\Gamma_s \vdash \ }}
\newcommand{\envEC}{{\Gamma_s , x:T_1 \vdash \ }}
\newcommand{\tyC}{{\vdash_c \ }}
\begin{document}

\begin{table}
\begin{center}
\begin{tabular}{|l c l r|}
\hline
$\mathsf {Syntax}:$&&&\\
&&&\\
$t$&$::=$&&$\mathsf {terms}$\\
&&$x$&$\mathsf {variable}$\\
&&$\abs {x}{T}{t}$&\ $\mathsf {abstraction}$\\
&&$\app {t_1}{t_2}$&$\mathsf {application}$\\
&&&\\
$c$&&&$\mathsf {configurations}$\\
&&$\conf {x}$&\\
&&$\conf {\abs {x}{T}{t}}$&\\
&&$\conf {\app {t_1}{t_2}}$&\\
&&$c \ c $&\\
$v$&$::=$&&$\mathsf {values}$\\
&&$\conf {\abs {x}{T}{t}}$&$\mathsf {clause }$\\
&&&\\
$T$&$::=$&&$\mathsf {types}$\\
&&$T \to T$&$\mathsf {type \ of \ functions}$\\
&&&\\
$\Gamma$&$::=$&&$\mathsf {contexts}$\\
&&$\varnothing$&$\mathsf {empty \ context}$\\
&&$\Gamma , x:T$&$\mathsf {term \ variable \ binding}$\\
&&&\\
$[s]$&$::=$&&$\mathsf {explicit \ substitutions}$\\
&&$[ \ ]$&$\mathsf {empty \ substitutions}$\\
&&$\subxD$&$\mathsf {not \ empty \ substitutions}$\\
\hline
\end{tabular}
\caption{Syntax of the simply typed lambda-calculus with explicit substitution.}
\label{tabla:sencilla}
\end{center}
\end{table}

\begin{table}
\begin{center}
\hspace*{-1.8cm}
\begin{tabular}{|l|l|}
\hline
\begin{tabular}{l c r}
&&\\
\multicolumn{2}{l}{$\mathsf {Typing \ terms}:$}&$\env t:T$\\
&&\\
&$\inference {x:T \in \Gamma}{\env x:T}$&$\rulename{TVar}$\\
&&\\
&$\inference {\envE t_2:T_2}{\env \absD:T_1 \to T_2}$&$\rulename{TAbs}$\\
&&\\
&$\inference {\env t_1 : T_{11} \to T_{12} & \env t_2 : T_{11}}{\env \appD : T_{12}}$&$\rulename{TApp}$\\
&&\\
&&\\
&&\\
&&\\
\end{tabular}&
\begin{tabular}{l c r}
&&\\
\multicolumn{2}{l}{$\mathsf {Typing \ configurations}:$}&$\tyC c:T$\\
&&\\
&$\inference {\tyC s(x):T}{\tyC \conf {x}:T}$&$\rulename{TCVar}$\\
&&\\
&$\inference {\envEC t_2:T_2}{\tyC \conf {\absD} :T_1\to T_2}$&$\rulename{TCAbs}$\\
&&\\
&$\inference {\tyC \conf {t_1} : T_{11} \to T_{12} & \tyC \conf {t_2} : T_{11}}{\tyC  \conf {\appD} : T_{12}}$&$\rulename{TCApp}$\\
&&\\
&$\inference {\tyC \conf {c_1} : T_{11} \to T_{12} & \tyC \conf {c_2} : T_{11}}{\tyC  \conf {\app {c_1}{c_2}} : T_{12}}$&$\rulename{TCCApp}$\\
&&\\
\end{tabular}\\
\hline
\end{tabular}
\hspace*{-1cm}
\caption{Typing rules for terms and configurations.}
\label{tabla:sencillaA}
\end{center}
\end{table}

\begin{table}
\begin{center}
\begin{tabular}{|c l|}
\hline
&\\
$\confx {x} \tto v$&$\rulename{VarOk}  $\\
&\\
$\confy {x} \tto x \ [s]$&$\rulename{VarFail}  $\\
&\\
$\appD \ [s] \tto t_1 [s] \ t_2 [s]$&$\rulename{AppSub} $\\
&\\
$\absD \ [s] \ v \tto \confx{t_2}  $&$\rulename{\beta}  $\\
&\\
$\inference {c_1 \tto c_1'}{c_1 \ c_2 \tto c_1' \ c_2} $&$\rulename{\nu}  $\\
&\\
$ \inference {c \tto c'}{v  \ c \tto v \ c'}$&$\rulename{\mu}  $\\
&\\
\hline
\end{tabular}
\caption{Evaluation Rules.}
\label{tabla:sencilla}
\end{center}
\end{table}
LEMMA [Inversion of the Typing Relation of terms]:
\begin{enumerate}
\item If $\env x : R$ , then $x : R \in \Gamma$
\item If $\env \absD : R$, then $R = T_1 \to R_2$ for some $R_2$, with $\envE t_2 : R_2$.
\item If $\env t_1 \ t_2 : R$, then there is some type $T_{11}$ such that $\env t_1 : T_{11} \to R$ and $\env t_2 : T_{11}$.
\end{enumerate}
$Proof:$ Immediate from the definition of the typing relation.\ \\

LEMMA [Inversion of the Typing Relation of configurations]:
\begin{enumerate}
\item If $\tyC x \ [s] : R$ , then $\tyC s(x) : R$
\item If $\tyC \absD \ [s]: R$, then $R = T_1 \to R_2$ for some $R_2$, with $\envEC t_2 : R_2$.
\item If $\tyC  t_1 \ t_2\ [s] : R$, then there is some type $T_{11}$ such that $\tyC t_1 [s] : T_{11} \to R$ and $\tyC t_2 [s] : T_{11}$.
\item If $\tyC   c_1 \ c_2 : R$, then there is some type $T_{11}$ such that $\tyC c_1 : T_{11} \to R$ and $\tyC c_2: T_{11}$.
\end{enumerate}
$Proof:$ Immediate from the definition of the typing relation.\ \\

LEMMA [Canonical Forms]:
\begin{enumerate}
\item If $v$ is a value of type $T_1 \to T_2$, then $v = \absD \ [s]$.
\end{enumerate}
$Proof:$ Straightforward.\ \\

DEFINITION: The typing context built from a substitution $[s]$, writing $\Gamma_s$, it is defined as follows:
\[ \Gamma_s = \begin{cases} 
     \varnothing & [s] = [ \ ]\\
      \Gamma_{s'}, x:typeof(v) & [s] = \subx 
   \end{cases}
\]
where $typeof(v) = T$ if $\tyC v : T$. \ \\

THEOREM [Progress]: Suppose $c$ is a well-typed configuration (that is, $ \tyC c : T$ for some $T$). Then either $c$ is a value or else there is some $c'$ with $c \tto c'$. \ \\
Proof: Straightforward induction on typing derivations.
\begin{enumerate}
\item $Case \ TCVar$ : $c = x[s]$, with $\tyC s(x):T$. Since $x \in dom([s])$, if the substitution $[s] = \subx$, then rule $VarOk$, applies, otherwise, rule $VarFail$ applies.

\item $Case \ TCAbs$: $c = \absD \ [s]$. This case is immediate, since clause are values.

\item $Case \ TCApp$: $c = t_1 \ t_2 \ [s]$, so rule $AppSub$ applies to $c$.

\item $Case \ TCCApp$: $c = c_1 \ c_2$, with $\tyC c_1 : T_{11} \to T$, for some $T_{11}$ and $\tyC c_2 : T_{11}$, for the inversion lemma. Then, by the induction hypothesis, either $c_1$ is a value or else it can make a step of evaluation, and likewise $c_2$. If $c_1$ can take a step, then rule $\nu$ applies to $c$. If $c_1$ is a value and $c_2$ can take a step, then rule $\mu$ applies. Finally, if both $c_1$ and $c_2$ are values, then the canonical forms lemma tells us that $c_1$ has the form $\lambda x: T_{11}.t_{12} \ [s]$, and so rule $\beta$ applies to $c$.
\end{enumerate}

DEFINITION: A substitution $[s]$ is said well typed with a typing context $\Gamma$, writing $\env [s]$, if $dom([s]) = dom(\Gamma)$ and $ \tyC s(x):\Gamma(x)$, for every $x \in dom(\Gamma)$. \ \\

LEMMA[Permutation]: If $\env \ t : T$ and $\Delta$ is a permutation of $\Gamma$, then $\Delta \vdash \ t : T$.\\
$Proof$: Straightforward induction on typing derivations.\ \\

LEMMA[1]: If $\env [s]$ then $Gamma$ is a permutation of $\Gamma_s$.\\
$Proof$: Straightforward by the definition of well typed substitution.\ \\

LEMMA[2]: If $\env [s]$ and $\tyC v : T$, then $\envE \subx$. \\
$Proof$: Straightforward by the definition of well typed substitution.\ \\

LEMMA[3]: If $\env [s]$ then $\tyC t \ [s] : T$ if and only if $\env t :T$. \\
$Proof$: Straightforward induction on typing derivations.\ \\

THEOREM[Preservation]: If $\tyC c : T$ and $c \tto c'$, then $\tyC t' : T$.\\
$Proof$: By induction on a derivation of $t : T$.

\begin{enumerate}
\item $Case \ TCVar$: $c = x \ [s]$, with $\tyC s(x):T$. If the last rule in the derivation is $TCVar$, then we know from the form of this rule that $c$ must have the form $x \ [s]$, for some $x$. Now, looking at the evaluation rules with this form on the left-hand side, we find that there are two rule by which $c \tto c'$ can be derived: $VarOk$ and $VarFail$. We consider each case separately.
\begin{itemize}
\item $Subcase \ VarOk$: $[s] = \subx$ and $c' = v$. Since $s(x) = v$, then  $\tyC v : T$, to conclude that $\tyC c' : T$.

\item $Subcase \ VarFail$: $[s] = \suby$ and $c' = x \ [s']$. Since $s(x) = s'(x)$, then $\tyC s'(x) : T$, to conclude that $\tyC x \ [s']:T $, that is $\tyC c' : T$. 
\end{itemize}


\item $Case \ TAbs$: $c = \absD \ [s]$. It cannot be the case that $c \tto c'$, because $c$ is a value, then the requirements of the theorem are vacuously satisfied. 

\item $Case \ TCApp$: $c = t_1 \ t_2 \ [s]$, $\tyC t_1 \ [s]: T_2 \to T$ and $\tyC t_2 \ [s]: T_2$. If the last rule in the derivation is $TCApp$, then we know from the form of this rule that $c$ must have the form $t_1 \ t_2 \ [s]$, for some $t_1$ and $t_2$. Now, looking at the evaluation rules with this form on the left-hand side, we find that there are only one rule by which $c \tto c'$ can be derived: $AppSub$. With this rule $c' = t_1[s] \ t_2[s]$, then we can apply the rule $TCCApp$, to conclude that $\tyC t_1[s] \ t_2[s]: T$ , that is $\tyC c' : T$. 

\item $Case \ TCCApp$: $c = c_1 \ c_2$, $\tyC \ c_1 : T_2 \to T$ and $\tyC c_2 : T_2$. If the last rule in the derivation is $TCCApp$, then we know for the form of this rule that $c$ must have the form $c_1 \ c_2$, for some $c_1$ and $c_2$. Now, looking at the evaluation rules with this form on the left-hand side, we find that there are three rules by which $c \tto c'$ can be derived: $\nu$, $\mu$ and $\beta$. We consider each case separately.
\begin{itemize}

\item $Subcase \ \nu$: $c_1 \tto c_1', \ c' = c_1' \ c_2$. By the induction hypothesis, $\tyC c_1' : T_2 \to T$, then we can apply rule $TCCApp$, to conclude that $\tyC c_1' \ c_2: T$ , that is $\tyC c' : T$.

\item $Subcase \ \mu$: $c_2 \tto c_2', \ c' = c_1 \ c_2'$. By the induction hypothesis, $\tyC c_2' : T_2 $, then we can apply rule $TCCApp$, to conclude that $\tyC c_1 \ c_2' : T$, that is $\tyC c' : T$.

\item $Subcase \ \beta$: $c_1 = \lambda x:T_{1}.t_{12}\ [s]$, $c_2 = v$, $c' = \confxE{t_{12}}$ and $\envEC t_{12}: T$ for the inversion lemma. Since we know that $\envEC \subx$, the resulting configuration $\tyC \confxE {t_{12}}':T$, for the [LEMMA 3], that is $\vdash c' : T$.
\end{itemize}
\end{enumerate}
\end{document}
