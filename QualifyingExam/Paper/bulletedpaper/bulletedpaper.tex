\documentclass{article}

%%%% Packages
\usepackage{hyperref} %package for generating bookmarks
%\usepackage[utf8]{inputenc}
%\usepackage[T1]{fontenc}
\usepackage{graphicx}
%used for figure, for stopping auto float
\usepackage{float}
\usepackage{fancyhdr}
\usepackage{amsmath}
\usepackage{amssymb}
\usepackage{stackengine}
\usepackage{graphicx}
\usepackage{verbatim}
\graphicspath{ {img/} }

%Multiline comment
\usepackage{verbatim}
%for logic proofs
\usepackage{proof}
\usepackage{pdflscape}
% Package that contains captionof
\usepackage{caption}
% Package for code listings
\usepackage{listings}
\lstset{language=ML} 

%\usepackage{color}
\usepackage[usenames, dvipsnames]{color}
%\setcounter{secnumdepth}{6}
\usepackage{authblk}

\title{\bf Overloading}
\author{Elizabeth Labrada Deniz
\thanks{Funded by grant CONICYT, CONICYT-PCHA/Doctorado Nacional/2015-63140148}}
\affil{Computer Science Department (DCC), University of Chile, Chile}
\date{}
\setcounter{Maxaffil}{0}
\renewcommand\Affilfont{\itshape\small}

\begin{document}
	\maketitle
	\renewcommand{\abstractname}{Abstract}	
	\begin{abstract}
	\end{abstract}	
\section{Introduction}
\begin{itemize}
    \item Que es Overloading?
    \item Para que sirve, nombrando algunos lenguajes conocidos que lo tienen.
    \item Explicar que es considerado un tipo de polimorfismo.
    \item Explicar brevemente distintos tipos de polimorfismo y las diferencias con overloading~\cite{CardelliWegner}.
	\item Explicar que la mayoria de los lenguajes que poseen polimorfismo resuelven el overloading en presencia de otras clases de polimorfismo, por ejemplo type classes y Java. Queremos iniciar con un lenguaje que solo contenga  overlaoding, para estudiar esta caracteristica en si.\cite{Nipkow:1993:TCT:158511.158698, Odersky:1995:SLO:224164.224195, Ribeiro2013, wadlerBlott:popl89}
	\item Explicar brevemente como se resuelve overloading en la mayoria de los lenguajes, que es en tiempo de compilacion, o al menos se utiliza informacion en tiempo de compilacion, tal como FeatherWeight Java con multi-method. Nombrar Common Lisp como un lenguaje diferente.
	\item Decir que el lenguaje que queremos dise\~nar es para posteriormente estudiarlo con tipos graduales, y asi explicar que nos conllevo al diseño de la semantica del lenguaje.
                                                                                                                                                                                                                                                                                                                                                                                                                                                                                                                                                                                                                                                                                                                                                                                                                                                                                                                                                                                                                                                                                                                                                                                                                                                                                                                                                                                                                                                                                                                                                                                                                                                                                                                                                                                                                                                                                                                                                                                                                                                                                                                                                                                                                                                                                                                                                                                                                                                                                                                                                                                                                                                                                                                                                                                                                                                                                                                                                                                                                                                                                                                                                                                                                                                                                                                                                                                                                                                                                                                            
\end{itemize}
\begin{figure}
\begin{center}
\[ Polyphormism = 
\begin{cases} 
     universal & 
     \begin{cases} 
     	parametric &\\
     	inclusion &
   	\end{cases} \\
		 & \\
     ad \ hoc & 
     \begin{cases} 
     	overloading & \\
     	coercion &
   	\end{cases}
 \end{cases}
\]
\caption{Varieties of polyphormism.}
\label{figure:varpolyphormism}
\end{center}
\end{figure}



\section{Background}\label{section:concepts}\
\begin{itemize}
\item Substitucion explicita, explicar brevemnte que es y porque la vamos a utilizar.
\item Ambiguedad 
\item Local Type Inference, si se utiliza en el sistema de tipo
\item Otros
\end{itemize}
\section{Calculus}\label{section:polymorphism}
\begin{itemize}
\item Explicar que agrega cada semantica con respecto a la anterior, por ejemplo deteccion de errores. 
\item Explicar relaciones entre las semanticas,

\end{itemize}

\subsection{Semantic 1}
\subsection{Semantic 2}
\subsection{Semantic 3}
\subsection{Semantic 4}
\subsection{Type System}
\subsection{Proofs}
\section{Conclusions}
\begin{itemize}
\item Conclusiones
\item Trabajo Futuro
\end{itemize}
%Bibliography
\medskip 
\bibliographystyle{abbrv}
\bibliography{gsta,pleiad,gp,bib,common}
\end{document}
